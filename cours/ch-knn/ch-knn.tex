\chapter{L'algorithme des k plus proches voisins}
Cet algorithme s'appelle $k$ \textit{Nearest Neighbors} en anglais, nous l'appellerons donc $k$NN.

\section{Des questions concrètes}

\begin{enumerate}
	\item Une entreprise de vente d'articles en lignes collecte les informations de ses clients. Elle en a accumulé un grand nombre, tels que 
	\begin{itemize}
		\item l'âge ;
		\item le sexe ;
    	\item l'adresse ;
    	\item les revenus moyens mensuels ;
    	\item \textit{et cætera}.
	\end{itemize}
    En fonction des achats, réguliers ou non, elle a placé ses client.e.s dans des catégories telles que
    \begin{itemize}
    	\item client\cdot e fidèle ;
    	\item client\cdot e à fidéliser ;
    	\item \textit{et cæetera}.
    \end{itemize}
    Un nouveau client se présente.\\
    Connaissant son âge, son sexe, son adresse et ses revenus mensuels, dans quelle catégorie l'entreprise va-t-elle le placer ?

	\item On a mesuré la largeur et la longueur des pétales et des  sépales d'iris de 3 catégories (\textit{iris setosa}, \textit{iris versicolor} et \textit{iris virginica}) 
    On effectue des mesures sur une nouvelle fleur. Dans quelle catégorie va-t-on la placer ? 

\end{enumerate}

Pour répondre aux deux questions précédentes on utilise le même algorithme : $k$NN.


\section{Un algorithme pour y répondre}

\floatpictureleft{0.33}{ch-knn/img/plot}{On considère deux nuages de points, l'un vert et l'autre bleu.
Le point rouge (appelons-le P) doit-il être considéré comme appartenant au nuage vert ou au nuage bleu ?\\

Pour répondre à cette question, on va }\medskip\par



\begin{methode}[ : kNN]
\begin{itemize}
	\item choisir un entier $k$ impair (3, 5 ou 7 typiquement) ;
	\item calculer les distances entre P et tous les autres points ;
	\item sélectionner les $k$ points les plus proches de P ;	
	\item regarder leurs couleurs.
\end{itemize}
Puisque $k$ est impair il n'y aura pas de situation d'\textit{ex æquo} et, suivant la couleur majoritaire des « $k$ plus proches voisins » de P, on pourra choisir celle de P.\\

C'est cela, l'algorithme $k$NN.
\end{methode}

La seule chose qui change selon la situation, c'est la \textit{distance} que l'on utilise. Avec les nuages de points, on utilise la distance euclidienne :
$$d(A,B)=\sqrt{\left(x_B-x_A\right)^2+\left(y_B-y_A\right)^2}$$
Mais on peut utiliser une distance différente suivant la situation.\\
Par exemple, si on ne manipule non plus des points mais des quadruplets $A=(a_0,a_1,a_2,a_3)$, alors on peut poser

$$d(A,B)=\sqrt{(b_0-a_0)^2+(b_1-a_1)^2+(b_2-a_2)^2+(b_3-a_3)^2}$$

ou encore

$$d(A,B)=|b_0-a_0|+|b_1-a_1|+|b_2-a_2|+|b_3-a_3| $$

Et d'ailleurs, si c'est la proximité de la première composante qui importe le plus, on peut définir
$$d(A,B) =50 \times |b_0-a_0|+|b_1-a_1|+|b_2-a_2|+|b_3-a_3|$$
Pour plus de renseignements sur ce qu'est une distance, tu peux consulter Wikipédia.

