\documentclass{nsibook}
\begin{document}
\chapter{Variables et affectations}

\section{Le symbole =}

En mathématiques, le symbole $=$ a plusieurs significations
\begin{itemize}
	\item dans $2+2=4$, on peut comprendre $=$ comme un opérateur d'évaluation : $2+2$, cela « donne » $4$ ;
	\item dans $\mathcal{P}=2\times(\ell+L)$, on peut considérer que $=$ sert à définir ce qu'est le périmètre d'un rectangle de dimensions $\ell$ et $L$ ; 
	\item dans $3x + 2 = 4x +5$, le $=$ sert à convenir que les 2 membres ont la même valeur et on cherche s'il existe un ou des nombres $x$ qui satisfont l'égalité (appelée équation) ;
	\item \textit{et cætera}. 
\end{itemize}

En \textsc{Python}, le symbole \mintinline{python}{=}  n'a qu'un seul sens : il sert à l'\textit{affectation}.

\section{L'affectation}
Il s'agit de « stocker » une valeur dans un endroit de la mémoire auquel \textsc{Python} donne un nom\footnote{En réalité c'est plus compliqué mais cela ne nous intéresse pas.}. Voici un exemple d'affectation : 
\begin{center}\Large
\mintinline{python}{a = 2}
\end{center}\
\begin{itemize}
\item   \mintinline{python}{2} est une \textit{valeur} de type \mintinline{python}{int} ;
\item   la \textit{variable} \mintinline{python}{a} est créée ;\
\item   \mintinline{python}{a} est « attachée » à la valeur \mintinline{python}{2} ;
\item   par extension \mintinline{python}{a} est également de type \mintinline{python}{int}.
\end{itemize}
Au cours d'un programme la valeur associée à une variable peut changer... D'où le nom de \textit{variable}.\\

Que fait le programme suivant ?

\begin{pyc}
	\begin{minted}{python}
		x = 0
		x = x + 1
		print(x)    
	\end{minted}
\end{pyc}

\begin{itemize}
	\item il crée une variable \mintinline{python}{x} de type \mintinline{python}{int} valant \mintinline{python}{0} ;
    \item il évalue \mintinline{python}{x + 1}, trouve \mintinline{python}{1} et affecte cette valeur à \mintinline{python}{x} ;
    \item évalue \mintinline{python}{x}, trouve 1 et donc affiche \mintinline{python}{1}.
\end{itemize}

\begin{aretenir}
	En mathématiques, $x = x + 1$ est une équation sans solution.\\

	En \textsc{Python}, l'instruction \mintinline{python}{x = x + 1} sert à augmenter la valeur de \mintinline{python}{x} de 1 (on dit aussi \textit{incrémenter}).
\end{aretenir}
\end{document}