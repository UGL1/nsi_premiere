\chapter{Recherche dichotomique}
\introduction{Plus petit ou plus grand ?}

\section{Présentation de l'algorithme}
Lorsqu'on dispose d'une liste d'éléments et d'un élément particulier, on peut rechercher si cet élément appartient ou non à la liste : il suffit de parcourir les éléments de la liste un par un et de les comparer à l'élément que l'on cherche. Cette démarche peut être améliorée si la liste possède des propriétés particulières, notamment si c'est une liste d'entiers triée.\\

On veut écrire une fonction \texttt{recherche\_dichotomique} qui :
\begin{itemize}
    \item   En entrée prend \begin{itemize}
                                \item   une liste \mintinline{python}{liste_triee} de $n$ entiers \textit{triée dans l'ordre croissant};
                                \item   un entier \mintinline{python}{val}.
                            \end{itemize}
    \item  Renvoie \begin{itemize}
                        \item   l'indice de \mintinline{python}{val} dans \mintinline{python}{liste_triee} si \mintinline{python}{val} appartient à \mintinline{python}{liste_triee};
                        \item   -1 si a n'appartient pas à \mintinline{python}{liste_triee}.
                    \end{itemize} 
\end{itemize}

\begin{exemple}[]
\begin{itemize}
    \item   \mintinline{python}{recherche_dichotomique([11, 20, 32, 33, 54], 32)} \\renvoie 2 car 32 est l'élément d'indice 2 de la liste.
    \item   \mintinline{python}{recherche_dichotomique([20, 32, 33, 54], 40)} \\renvoie -1 car 40 ne figure pas dans la liste.
\end{itemize}
\end{exemple}

\begin{methode}[]
On compare \mintinline{python}{val} avec l'élément $m$ qui se situe \og à peu près au milieu de \mintinline{python}{liste_triee}\fg{}.
\begin{itemize}
    \item   si $val$ est égal à $m$ c'est gagné, on renvoie l'indice de $m$ dans  \mintinline{python}{liste_triee};
    \item   sinon si $val>m$ on recommence avec la liste des éléments situés après $m$.
    \item   sinon c'est que $val<m$ et on recommence avec la liste des éléments situés avant $m$.
\end{itemize}
On itère ce procédé tant que la liste des valeurs à examiner n'est pas vide. Si on arrive à une liste vide c'est que \mintinline{python}{val} n'est pas dans \mintinline{python}{liste_triee}
\end{methode}
Voici l'algorithme traduit en \textsc{Python} :
\begin{pys}
\begin{minted}{python}
def recherche_dichotomique(liste_triee, val):
    gauche = 0  # début de la plage de valeurs à regarder
    droite = len(liste_triee) - 1  # fin de la plage
    while gauche <= droite:  # tant que la plage est non vide
        milieu = (gauche + droite) // 2  # on prend grosso modo le milieu
        if liste_triee[milieu] == val:
            return milieu  # si on trouve val au milieu c'est gagné
        elif liste_triee[milieu] > val:  # si on a dépassé val
            droite = milieu - 1  # alors on regarde avant
        else:
            gauche = milieu + 1  # sinon on regarde après
    # si on est sorti de la boucle
    return -1  # c'est qu'on n'a pas trouvé val
\end{minted}
\end{pys}

\section{\'Etude de l'algorithme}

Trois question se posent :

\begin{enumerate}
    \item   Pourquoi la boucle \textit{tant que} s'arrête-t-elle toujours ?\\ On dit que c'est un problème de \textit{terminaison}.
    \item   Quand la fonction renvoie -1, est-ce que cela veut bien dire que \mintinline{python}{val} n'est pas dans \mintinline{python}{liste_triee} ? De même quand la fonction renvoie une valeur $m\neq -1$, est-ce que cela veut bien dire que \mintinline{python}{val==liste_triee[m]} ?\\
    C'est un problème de \textit{correction}.
    \item   Pourquoi cette fonction est-elle plus rapide qu'un parcours des éléments un par un ?\\ C'est un problème de \textit{complexité}
\end{enumerate}

\section{Terminaison}

Pour prouver qu'une boucle \textit{tant que} se termine, on détermine un \textit{variant} de boucle.

\begin{definition}[]
Un variant de boucle est un \textit{entier positif qui décroît strictement à chaque itération de boucle}. On le choisit de sorte à ce que lorsqu'il atteint zéro (ou un, en tout cas une petite valeur) la boucle se termine.\\
\end{definition}

Dans notre cas, le variant de boucle est l'entier \mintinline{python}{v} défini par \mintinline{python}{v = droite-gauche} :
\begin{itemize}
    \item   la condition du \mintinline{python}{while} est liée à \mintinline{python}{v} puisqu'elle se réécrit \mintinline{python}{while v >= 0};
    \item   quand on rentre dans la boucle on définit \mintinline{python}{milieu} et on a toujours \\
    \mintinline{python}{gauche <= milieu <= droite};
    \item   si \mintinline{python}{ liste_triee[milieu] == val} alors la boucle s'arrête.
        
    \item   sinon si \mintinline{python}{ liste_triee[milieu] > val} alors :\\
            \mintinline{python}{droite = milieu - 1} et donc\\
            \mintinline{python}{v} devient (appelons-le \texttt{v2} temporairement) \mintinline{python}{v2 = milieu - 1 - gauche} donc\\
            \mintinline{python}{v2 < milieu - gauche} et puisque \mintinline{python}{milieu <= droite} on a\\
            \mintinline{python}{v2 < droite - gauche} et ainsi \mintinline{python}{v2 < v}.
    \item   sinon \mintinline{python}{ liste_triee[milieu] < val} et à ce moment :\\
            \mintinline{python}{gauche = milieu + 1} et donc\\
            \mintinline{python}{v} devient (appelons-le \texttt{v2} temporairement) \mintinline{python}{v2 = droite - (milieu+1)} donc\\
            \mintinline{python}{v2 < droite - milieu} et puisque \mintinline{python}{gauche <= milieu } on a\\
            \mintinline{python}{v2 < droite - gauche} et ainsi \mintinline{python}{v2 < v}.         
\end{itemize}

Ainsi les valeurs de v décroissent strictement, donc finissent (si on ne trouve pas \texttt{val}) par atteindre zéro et la boucle se termine.\\
On dit qu'on a prouvé la \textit{terminaison} de la fonction.

\section{Correction}

Pour prouver que cette fonction est correcte, on doit utiliser un \textit{invariant de boucle} (ne pas confondre avec le \textit{variant de boucle}).

\begin{definition}[]
Un \textit{invariant de boucle} est une propriété P dépendant éventuellement des variables du programme
\begin{itemize}
    \item   P doit être vraie avant l'entrée dans la boucle \textit{tant que};
    \item   P doit rester vraie à chaque itération de boucle;
    \item   à la fin de la boucle, P doit nous permettre de conclure que la fonction \og fait bien ce qu'elle doit faire\fg{}.
\end{itemize}
\end{definition}


Dans notre cas voici l'invariant de boucle :
\begin{center}
P : \og si \mintinline{python}{val} est dans \mintinline{python}{liste_triee} son indice ne peut-être qu'entre \mintinline{python}{gauche} et \mintinline{python}{droite}\fg{}
\end{center}

\begin{itemize}
    \item   avant l'entrée dans la boucle \mintinline{python}{while}, on a\\
     \mintinline{python}{gauche = 0} et \mintinline{python}{droite = len(liste_triee)-1} donc P est trivialement vérifiée;
    \item   dans la boucle, si  \mintinline{python}{liste_triee[milieu] == val} alors on renvoie \mintinline{python}{val} et la fonction s'arrête et donne bien le résultat attendu;
    \item   sinon si \mintinline{python}{liste_triee[milieu] > val} alors puisque la liste est triée, la position de \mintinline{python}{val} ne peut être qu'entre \mintinline{python}{gauche} et \mintinline{python}{milieu-1}, or \mintinline{python}{droite} est actualisée avec cette valeur, et P reste vraie.
    \item   de même si \mintinline{python}{liste_triee[milieu] < val}
\end{itemize}
En sortie de boucle on n'a plus la condition \\
\mintinline{python}{gauche <= droite} donc on a \mintinline{python}{gauche > droite}. Supposons que \mintinline{python}{val} appartienne à la liste, alors puisque l'invariant de boucle P est vrai y compris à la fin de la boucle, cela veut dire que \mintinline{python}{val} doit être entre \mintinline{python}{gauche} et \mintinline{python}{droite} mais c'est absurde puisque\\  \mintinline{python}{gauche > droite}. Donc notre supposition est fausse : \mintinline{python}{val} n'appartient pas à la liste\footnote{Ceci s'appelle un \textit{raisonnement par l'absurde.}}.\\

On a donc prouvé la correction de notre fonction.

\section{Complexité}

On va ici évaluer le nombre d'étapes nécessaires au déroulement de la fonction. On va raisonner dans le pire des cas : \mintinline{python}{val} n'appartient pas à la liste.

Sans détailler les calculs\footnote{Peux-tu le faire ?}, à chaque itération de boucle, \mintinline{python}{droite-gauche+1} (nombre de valeurs qui restent à examiner) est au moins divisé par 2 et lorsque  cette valeur vaut 1, c'est qu'on est à la dernière itération de boucle : \mintinline{python}{droite == gauche} et on est sûr ou bien de trouver \mintinline{python}{val} à cet endroit, ou bien on sort de la boucle.\\

Ainsi, pour une liste de longueur 2 on est sûrs d'arriver au résultat en 2 itérations, pour une liste de longueur 4, en 3 itérations et en généralisant, si la liste est de longueur $2^n$, en $n+1$ itérations.\\

Par exemple pour un tableau de longueur 1000, puisque $2^9<1000<2^{10}$, on est sûr d'arriver au résultat au plus en 10 itérations.\\

\begin{definition}[]
Soit $n$ un entier naturel non nul, on appelle \textit{logarithme en base 2} de $n$ l'unique réel $x$ solution de $$2^x=n$$
Ce nombre $x$ est noté $\log_2(n)$.

% Le nombre de bits nécessaires pour écrire $n$ en binaire est $$\lfloor\log_2 n\rfloor +1$$
\end{definition}

Ce que l'on vient de prouver, c'est que pour une liste de taille $n$, la fonction\\
 \mintinline{python}{recherche_dichotomique} nécessitera au plus $E(\log_2(n))+1$ itérations pour déterminer si oui ou non une valeur appartient à cette liste ($E$ représente la fonction \textit{partie entière}).\\
 
\begin{propriete}
Soit une liste triée de longueur $n\in\N^*$.\\
Soit $p$ le nombre de bits nécessaires pour écrire $n$ en base 2.\\

La recherche dichotomique d'une valeur dans la liste nécessite \textbf{au plus} $p$ accès à cette liste.
\end{propriete}

Pour cette raison la complexité de l'algorithme de recherche dichotomique est dite \textit{logarithmique}. C'est bien mieux que celle de la recherche simple.