\documentclass[10pt,a4paper]{nsibook}
\usepackage{circuitikz}
\begin{document}
\begin{titlepage}
    \begin{center}
        \includegraphics[width=12cm]{yin_yang_python.png}\\[2em]

        {\bigtitlefont \LARGE\color{gray} NSI1\\}

        {\titlefont\Large\color{gray} Spécialité numérique et sciences informatiques en classe de première\\[2em]}

        {\color{gray}\textbf{Lycée Rabelais\\ Saint Brieuc\\ 2023-2024}}
    \end{center}
\end{titlepage}
\part{Représentation des données}
\include{ch-bases.tex}
\include{ch-entiers.tex}
\include{ch-reels.tex}
%!TEX root = <cours.tex>
\chapter{Représentation du texte}
\introduction{Peux-tu décoder ce texte ?}

\section{Le code ASCII}


Pour représenter les caractères que nous utilisons pour écrire, on a historiquement choisi d'associer \textit{un numéro} (ou code) à chacun de
ces caractères. La correspondance entre chaque caractère et son code était appelée un \textit{Charset}.\\
Puisqu'à l'origine seul un petit nombre de caractères était utilisé (les caractères de base anglo-saxons), un octet suffisait pour les
représenter tous.\\
Le fait de représenter en machine un jeu de caractères s'appelle réaliser un encodage (\textit{encoding} en Anglais).\\

Le premier encodage utilisé fut l'\textsc{ASCII}, qui signifie  \textit{American Standard Code for Information Interchange}.\\

\begin{center}
    \includegraphics[width=\columnwidth]{img/ASCII.png}\\
    \textit{La table ASCII}
\end{center}
Le code ASCII se base sur un tableau contenant les caractères les plus utilisés en langue anglaise : les lettres de l'alphabet en majuscule (de A
à Z) et en minuscule (de a à z),
les dix chiffres arabes (de 0 à 9), des signes de ponctuation (point, virgule, point-virgule, deux points, points d'exclamation et
d'interrogation, apostrophe ou \textit{quote}, guillemet
ou \textit{double quotes}, parenthèses, crochets etc.), quelques symboles et certains caractères spéciaux invisibles (espace, retour-chariot,
tabulation, retour-arrière, etc.).\\

\begin{exercice}
    Combien y a -t-il de caractères dans ce catalogue ?\\
    Combien de bits sont nécessaires pour pouvoir représenter tous leurs numéros de code ?
\end{exercice}


Pour représenter ces symboles ASCII, les ordinateurs utilisaient des cases mémoires de un octet, mais ils réservaient toujours le huitième bit pour le contrôle de parité : c'est un procédé de sécurité pour
éviter les erreurs, qui étaient très fréquentes dans les premières mémoires électroniques.\\
\begin{methode}[ : contrôle d'erreur par parité]
    \begin{itemize}
        \item 	On dispose d'un mot de 7 bits, par exemple 111 0011
        \item 	On compte le nombre de bits à 1, il y en a 5.
        \item 	On rajoute le bit de poids fort à 1 pour qu'en tout, il y ait toujours \textit{un nombre pair} de bits à 1.\\
              On obtient \boxed{1}111 0011, ce bit de poids fort jouant le rôle de \textit{code correcteur}.
        \item 	Un autre exemple : 001 1110 est codé \boxed{0}001 1110.
    \end{itemize}
\end{methode}

\begin{exercice}[]
    Voici un message reçu à l'issue d'une transmission :
    \begin{center}
        \texttt{53 E1 6C F5 70}
    \end{center}
    Ces 6 octets sont censés représenter 6 caractères ASCII, codées sur 7 bits le 8\eme étant réservé au contrôle d'erreur par parité.
    \begin{enumerate}
        \item 	Décoder ces 6 octets en disant s'il y a des erreurs ou non.
        \item 	Quel était le message initial ?
    \end{enumerate}
\end{exercice}


\section{L'insuffisance de l'ASCII}

Pour coder les lettres accentuées, inutilisées en Anglais mais très fréquentes dans d'autres langues (notamment le Français), on a décidé
d'étendre
le codage des caractères au huitième bit (les erreurs-mémoire étant devenues plus rares et les méthodes de contrôle d'erreurs plus efficaces).\\


\begin{exercice}[]
    Combien de nouveaux symboles a-t-on pu coder en autorisant le huitième bit dans le codage ?
\end{exercice}


On a alors pu coder toutes ces lettres et ainsi que de nouveaux caractères typographiques utiles tels que différents tirets.\\


\section{Le problème}

Le fait d'utiliser un bit supplémentaire a bien entendu ouvert des possibilités mais malheureusement les caractères de toutes les langues ne
pouvaient être pris en charge
en même temps.\\
La norme ISO 8859–1 appelée aussi Latin-1 ou Europe occidentale est la première partie d'une norme plus complète appelée \textbf{ISO 8859} (qui
comprend 16 parties)
et qui permet de coder tous les caractères des langues européennes. Cette norme ISO 8859–1 permet de coder 191 caractères de l'alphabet latin qui
avaient à
l'époque été jugés essentiels dans l'écriture, mais omet quelques caractères fort utiles (ainsi, la ligature œ n'y figure pas).\\

Dans les pays occidentaux, cette norme est utilisée par de nombreux systèmes d'exploitation, dont Linux et Windows. Elle a donné lieu à quelques
extensions
et adaptations, dont \textbf{Windows-12527} (appelée \textbf{ANSI}) et ISO 8859-158 (qui prend en compte le symbole €\  créé après la norme
ISO 8859-1).
C'est une source de grande confusion pour les développeurs de programmes informatiques car un même caractère peut être codé différemment suivant
la norme utilisée .\\
Voici les tableaux décrivant deux encodages :

\includegraphics[width=\columnwidth]{img/W1252andISO}\\


\begin{exercice}[]
    Ces deux encodages sont-ils totalement compatibles ? Pourquoi ?
\end{exercice}

\section{La multiplicité des encodages}

Au fil du temps une multitude d'encodages sont apparus, multipliant les sources de confusion.\\
Voici pour l'exemple une partie des encodages que \textsc{Python} reconnaît :
\alternaterowcolors
\begin{center}
    {\tiny
        \begin{tabular}{|c|c|c|}
            \hline
            \rowcolor{UGLiOrange}{\boxfont\color{white}Encodage} & {\boxfont\color{white}Alias Python}                                                    & {\boxfont\color{white}Langues concernées}                           \\
            \hline
            ascii             & 646,us-ascii                                                             & English                                               \\\hline
            big5              & big5-tw, csbig5                                                          & Traditional Chinese                                   \\\hline
            cp424             & EBCDIC-CP-HE, IBM424                                                     & Hebrew                                                \\\hline
            cp437             & 437, IBM437                                                              & English                                               \\\hline
            cp500             & EBCDIC-CP-BE, EBCDIC-CP-CH, IBM500                                       & Western Europe                                        \\\hline
            cp720             &                                                                          & Arabic                                                \\\hline
            cp737             &                                                                          & Greek                                                 \\\hline
            cp856             &                                                                          & Hebrew                                                \\\hline
            cp857             & 857, IBM857                                                              & Turkish                                               \\\hline
            cp864             & IBM864                                                                   & Arabic                                                \\\hline
            cp874             &                                                                          & Thai                                                  \\\hline
            cp932             & 932, ms932, mskanji, ms-kanji                                            & Japanese                                              \\\hline
            cp1251            & windows-1251                                                             & Bulgarian, Byelorussian, Macedonian, Russian, Serbian \\\hline
            cp1258            & windows-1258                                                             & Vietnamese                                            \\\hline
            euc\_kr           & euckr, korean, ksc5601, ks\_c-5601, ks\_c-5601-1987, ksx1001, ks\_x-1001 & Korean                                                \\\hline
            gbk               & 936, cp936, ms936                                                        & Unified Chinese                                       \\\hline
            latin\_1          & iso-8859-1, iso8859-1, 8859, cp819, latin, latin1, L1                    & West Europe                                           \\\hline
            iso8859\_14       & iso-8859-14, latin8, L8                                                  & Celtic languages                                      \\\hline
            koi8\_r           &                                                                          & Russian                                               \\\hline
            utf\_8            & U8, UTF, utf8                                                            & all languages                                         \\\hline
        \end{tabular}}
\end{center}

\begin{exercice}[]
    Lequel de ces encodages semble le plus performant ?
\end{exercice}


\section{L'Unicode}

La globalisation des échanges culturels et économiques a mis l'accent sur le fait que les langues européennes coexistent avec de nombreuses
autres
langues aux alphabets spécifiques voire sans alphabet (le Japonais utilise entre autres un syllabaire, chaque symbole représentant une syllabe).
La
généralisation de l'utilisation d'Internet dans le monde a ainsi nécessité une prise en compte d'un
nombre beaucoup plus important de caractères (à titre d'exemple, le mandarin possède à lui tout seul plus de 5000 caractères !).\\
Une autre motivation pour cette
évolution résidait dans les possibles confusions dues au trop faible nombre de caractères pris en compte ; ainsi, les symboles monétaires des
différents
pays n'étaient pas tous représentés dans le système ISO 8859-1, de sorte que les ordres de paiement internationaux transmis par courrier
électronique
risquaient d'être mal compris. La norme Unicode a donc été créée pour permettre le codage de textes écrits quel que soit le système d'écriture
utilisé.\\

Dans le système UTF-8, on attribue à chaque caractère un nom, une position normative et un bref descriptif qui seront les mêmes quelle que soit
la plate-forme informatique
ou le logiciel utilisés.\\
Un consortium composé d'informaticiens, de chercheurs, de linguistes et de personnalités représentant les états ainsi que les entreprises
s'occupe d'unifier toutes les pratiques en un seul et même système : \textit{l'Unicode}.\\

\begin{definition}
    L'\textit{Unicode}  est  une table de correspondance Caractère-Code (Charset), et l'\textit{UTF-8} est l'encodage correspondant (Encoding) le
    plus répandu.
\end{definition}

De nos jours, par défaut, les navigateurs Internet utilisent le codage UTF-8 et les concepteurs de sites pensent de plus en plus à créer leurs
pages web
en prenant en compte cette même norme ; c'est pourquoi il y a de moins en moins de problèmes de compatibilité : l'UTF-8 est aujourd'hui
majoritairement utilisé pour les sites du web, comme le montre ce graphique.

\begin{center}
    \includegraphics[width=10cm]{img/UTF8evol}
\end{center}

L'UTF-8 est également le codage le plus utilisé dans les systèmes GNU, Linux et compatibles pour gérer le plus simplement
possible des textes et leurs traductions dans tous les systèmes d'écritures et tous les alphabets du monde.\\

\section{L'UTF-8 côté technique}

La norme Unicode définit entre autres un ensemble (ou répertoire) de caractères. Chaque caractère est repéré dans cet ensemble par un index
entier aussi appelé « point de code ».\\
Par exemple le caractère « € » (euro) est le 8365$^{\text{ème}}$ caractère du répertoire Unicode, son index, ou point de code, est donc
8364
(on commence à compter à partir de 0).\\
Le répertoire Unicode peut contenir plus d'un million de caractères, ce qui est bien trop grand pour être codé par un seul octet (limité à des
valeurs entre 0 et 255).
La norme Unicode définit donc des méthodes standardisées pour coder et stocker cet index sous forme de séquence d'octets :
UTF-8 est la plus utilisée d'entre elles (il y a aussi des variantes comme UTF-16 et UTF-32).
En UTF-8, tout caractère est codé sur 1, 2, 3 ou 4 octets.\\
La principale caractéristique d'UTF-8 est qu'elle est \textit{rétro-compatible avec la norme ASCII}, c'est-à-dire que tout caractère ASCII se
code
en UTF-8 sous forme d'un unique octet, identique au code ASCII.\\
Par exemple « A » (A majuscule) a pour code ASCII 65 et se code en UTF-8 par l'octet 65, il en va de même pour tous les caractère ASCII.
Pour les autres, on procède comme ceci :
\begin{itemize}
    \item Chaque caractère est associé à son index Unicode.
    \item 	En général, cet index est exprimé en hexadécimal.
          Actuellement, presques toutes les valeurs de 0000 à FFFF, c'est-à-dire de 0 à 65535 sont
          attribuées à des \og alphabets \fg{} associés à des langues, des plus communes aux plus rares, et à divers symboles, tels que les
          symboles mathématiques.
          Au delà de FFFF on trouve des alphabets associés à des langues anciennes (cunéiformes, hiéroglyphes\ldots).
    \item  En fonction du nombre de bits nécessaires pour représenter en binaire  cet index, on utilise le codage suivant :\\

          \begin{center}
              {\scriptsize
                  \begin{tabular}{|c|c|c|}
                      \hline
                      \rowcolor{UGLiOrange}{\boxfont\color{white}Nombre de bits de l'index} & {\boxfont\color{white}Nombre d'octets pour coder en UTF-8} & {\boxfont\color{white}Schéma de codage}                                                           \\
                      \hline
                      de 0 à 7                  & 1                                   & 0xxx xxxx                                                                   \\
                      \hline
                      de 8 à 11                 & 2                                   & \textbf{110}x xxxx \textbf{10}xx xxxx                                       \\
                      \hline
                      de 12 à 16                & 3                                   & \textbf{1110} xxxx \textbf{10}xx xxxx \textbf{10}xx xxxx                    \\
                      \hline
                      de 17 à 21                & 4                                   & \textbf{1111 0}xxx \textbf{10}xx xxxx \textbf{10}xx xxxx \textbf{10}xx xxxx \\
                      \hline
                  \end{tabular} }
              \normalsize
          \end{center}
          \vspace{1em}
\end{itemize}
Par exemple le symbole €\  a un index Unicode qui vaut 8364.\\
\begin{itemize}
    \item 	8364 s'écrit 20AC en hexa, ce qui fait 10 0000 1010 1100 en binaire, soit 14 bits.\\
          On va donc utiliser 3 octets pour coder, conformément au schéma de codage.
    \item 	On commence par écrire le mot de 16 bits correspondant : \boxed{00}10 0000 1010 1100.\\
    \item 	On formate comme à la 3ème ligne du tableau : \\

          \textbf{1110} 0010 \textbf{10}00 0010 \textbf{10}10 1100\\

          Ce qui fait 3 octets : E2 82 AC en hexadécimal.\\

\end{itemize}


\begin{exercice}[]
    Si un ordinateur lit cet encodage UTF-8 du symbole €\  selon l'encodage ISO8859-15, qu'affichera-t-il ?\\

\end{exercice}


\section{Conclusion}

Même si l'encodage UTF-8 devient le standard international, certains développeurs, sites, ou applications en utilisent malgré tout encore
d'autres.\\

\begin{propriete}
    La notion de texte brut n'existe pas en informatique : lorsqu'un ordinateur lit un fichier texte il n'a \textit{a priori} aucun moyen de savoir
    quel est son encodage.
\end{propriete}

Beaucoup de documents indiquent donc en en tête leur format d'encodage : en HTML, on écrira dans l'en-tête d'une page:
\begin{html}
    \begin{minted}{html}
        <meta charset="utf8"/>
    \end{minted}
\end{html}

pour préciser qu'elle est encodée en UTF-8.\\
\section{Et Python dans tout ça ?}
En \textsc{Python}, on pourra aussi écrire : \mintinline{python}{# -*- coding: utf8 -*-} en première ligne de tout script pour signifier la même chose, \textit{et c\ae tera}.\\

\textsc{Python} gère très bien les encodages. On peut fabriquer un convertisseur très rapidement :

\begin{pyc}
    \begin{minted}{python}
fichier = open("nom_fichier", 'rt', encoding="utf8")
texte = fichier.read()
fichier.close()
    \end{minted}
\end{pyc}

Ceci permet de lire le contenu d'un fichier texte (d'où le \mintinline{python}{'rt'}, pour 'read text') d'un fichier texte encodé en UTF-8, et de le stocker dans la
variable \texttt{texte}, de type \mintinline{python}{str}.\\

\begin{pyc}
    \begin{minted}{python}
fichier = open("nom_fichier", 'wt', encoding="utf8")
fichier.write("Salut")
fichier.close()	
    \end{minted}
\end{pyc}

Permet d'écrire un fichier texte (d'où le \mintinline{python}{'wt'}, pour write text) en UTF-8.\\

\section{Exercices}
\begin{exercice}
    Analyser les deux fichiers \texttt{texte1(utf8).txt} et \texttt{texte1(utf8).txt} : ils sont tous les deux encodés en UTF-8.


    \begin{itemize}
        \item Ouvrir chaque fichier. Quelles sont les différences de contenu entre ces deux fichiers ?
        \item Faire un clic droit puis \texttt{Propriétés} comme ceci :\\

              \includegraphics[width=7cm]{img/expli1}\\

              Quelles sont les tailles de ces deux fichiers ?\\
              Comment, dans le détail, la différence de taille s'explique-t-elle ?

    \end{itemize}
\end{exercice}

\begin{exercice} Nous allons examiner des problèmes d'encodage.
    \begin{itemize}
        \item Ouvrir le fichier \texttt{index1(utf8).html} avec un navigateur, puis le fichier \texttt{index2(utf8).html} avec un navigateur.\\
              Que remarquez-vous ?\\

        \item 	Ouvrir ce dernier fichier avec un éditeur de texte (comme le bloc-notes Windows).\\
              D'où vient le problème ? Proposer une correction.\\
              Expliquer en détail pourquoi à la place des \og é \fg{}, il y a des \og é \fg{} .\\

        \item 	Ouvrir le fichier \texttt{index3(ISO8859-15).html} avec un navigateur.\\
              D'où vient le problème ? Proposer une correction.\\
              Expliquer en détail pourquoi il y a des \og points d'interrogation dans des losanges noirs \fg{}.\\
    \end{itemize}
\end{exercice}

\begin{exercice}
    \'Ecrire un script qui corrige le problème de  \texttt{index3(ISO8859-15).html} en le convertissant en UTF-8 (utiliser les scripts \textsc{Python} fournis à la section précédente).

\end{exercice}

\part{Architectures matérielles et \\systèmes d'exploitation}
\include{ch-turing.tex}
\include{ch-logique.tex}
\include{ch-systemes.tex}
\part{Programmation avec Python}
\chapter{Types de variables}
\label{ch:types}
\introduction{Chic type !}
\begin{aretenir}
    \begin{itemize}
        \item 	les types de variables simples sont \mintinline{python}{bool}, \mintinline{python}{int} et \mintinline{python}{float};
        \item 	les types \mintinline{python}{str} et \mintinline{python}{list} sont structurés : on accède à leurs éléments par des indices entiers;
        \item 	le type \mintinline{python}{dict} représente des couples \og clé-valeurs\fg{};
        \item 	il est parfois possible (et souhaitable) de convertir le type d'une variable.
    \end{itemize}
\end{aretenir}
\textsc{Python} distingue plusieurs types de variables, voici les principaux :
\section{Le type bool}
Il sert à représenter les \textit{variables booléennes} : une variable de type \mintinline{python}{bool} est une valeur logique et vaut \mintinline{python}{True} (vrai) ou
\mintinline{python}{False} (faux).


\begin{pys}\begin{minted}{python}
>>> a = bool()
>>> a
False
>>> b = True
>>> b
True
>>> c = (3 > 2)
>>> c
True
\end{minted}
\end{pys}

Les \mintinline{python}{bool} servent très souvent lorsqu'on veut \textit{tester} si une condition est vraie ou non (on y reviendra plus tard).\\
Sur les \mintinline{python}{bool}, on dispose d'\textit{opérations logiques} : \mintinline{python}{or} (ou), \mintinline{python}{and} (et) et \mintinline{python}{not} (non).

\begin{pys}\begin{minted}{python}
>>> True and False
False
>>> True or False
True
>>> not (3 < 1)
True
\end{minted}
\end{pys}

\section{Le type int}\label{sec:int}

Il sert à représenter les \textit{entiers relatifs} (\textit{integer} signifie \og entier \fg{} en Anglais).\\
Sur les \mintinline{python}{int}, on dispose des opérations \mintinline{python}{+} (addition), \mintinline{python}{-} (soustraction) et \mintinline{python}{*} (multiplication).

\begin{pys}\begin{minted}{python}
>>> a = int()
>>> a
0
>>> b = 3
>>> c = 2
>>> b + c
5
>>> 2 * b
6
>>> b - 2 * c
-1
\end{minted}
\end{pys}

On dispose également de \textit{deux opérations très pratiques} : soient \mintinline{python}{a} et \mintinline{python}{b} deux \mintinline{python}{int}, et \mintinline{python}{b} non nul, alors on peut effectuer
la \textit{division euclidienne} de \mintinline{python}{a} par \mintinline{python}{b}.\\
\mintinline{python}{a // b} est le \textit{quotient} et \mintinline{python}|a % b| est le \textit{reste}.

\begin{pys}\begin{minted}{python}
>>> 64 // 10
6
>>> 64 % 10
4
>>> 22 // 7
3
>>> 22 % 7
1
\end{minted}
\end{pys}



On dispose de l'opération d'\textit{exponentiation} (opération puissance), notée \mintinline{python}{**}.\\
\textbf{Attention :} Cette opération peut produire un résultat \textit{non-entier}, de type \mintinline{python}{float} (voir partie suivante).

\begin{pys}\begin{minted}{python}
>>> 2 ** 3
8
>>> 10 ** 4
10000
>>> 2 ** (-1)
0.5
\end{minted}
\end{pys}

Pour finir, la \textit{division décimale} peut être effectuée sur des entiers, mais elle renvoie un résultat de type \mintinline{python}{float}.

\begin{pys}\begin{minted}{python}
>>> 10 / 2
5.0
>>> 2 / 3
0.6666666666666666
\end{minted}
\end{pys}


\section{Le type str}

Le type \mintinline{python}{str} pour représenter les \textit{chaînes de caractères} (\mintinline{python}{str} est l'abréviation de \emph{string}, qui veut dire chaîne en anglais).
Lors d'une affectation à une variable de type \mintinline{python}{str}, on peut utiliser les symboles \mintinline{python}{'}, \mintinline{python}{"} ou même \mintinline{python}{'''} (suivant que la chaîne contient
des apostrophes, ou des guillemets).

\begin{pys}\begin{minted}{python}
>>> a = str()
>>> a
''
>>> b = 'Bonjour.'
>>> b
'Bonjour'
>>> 'J'aime Python.'
SyntaxError
>>> "J'aime Python."
"J'aime Python."
>>> "Je n'aime pas qu'on m'appelle "geek"."
SyntaxError
>>> """Je n'aime pas qu'on m'appelle "geek"."""
'Je n\'aime pas qu\'on m\'appelle "geek".'
\end{minted}
\end{pys}

L'opérateur + appliqué à deux chaînes de caractères produit une nouvelle chaîne qui est la concaténation (c'est-à-dire la mise bout-à-bout) des deux premières.\\

On peut aussi multiplier une variable de type \mintinline{python}{str} par un \mintinline{python}{int}.\\

\begin{pys}\begin{minted}{python}
>>> a = 'Yes'
>>> b = 'No'
>>> a + b
'YesNo'
>>> b + a
'NoYes'
>>> 2 * a + 3 * b
'YesYesNoNoNo'
\end{minted}
\end{pys}

Les variables de type \mintinline{python}{str} sont composées de caractères \emph{alphanumériques}.\\ On peut y accéder de la manière suivante :

\begin{pys}\begin{minted}{python}
>>> chaine = 'Bonjour !'
>>> chaine[0]
'B'
>>> chaine[5]
'u'
\end{minted}
\end{pys}

Voici comment \textsc{Python} représente  la chaîne précédente :

\begin{center}
    \alternaterowcolors
    \begin{tabular}{|c|c|c|c|c|c|c|c|c|c|}
        \hline
        i                        & 0 & 1 & 2 & 3 & 4 & 5 & 6 & 7 & 8 \\
        \hline
        \mintinline{python}{chaine[i]} & B & o & n & j & o & u & r &   & ! \\
        \hline
    \end{tabular}
\end{center}

On a parfois besoin de connaître la longueur (\emph{length} en anglais) d'une chaîne de caractères :

\begin{pys}\begin{minted}{python}
>>> chaine = 'onzelettres'
>>> len(chaine)
11
\end{minted}
\end{pys}

On peut aussi accéder facilement au dernier (ou à l'avant dernier) caractère d'une variable de type \mintinline{python}{str} :

\begin{pys}\begin{minted}{python}
>>> a = "M'enfin ?!"
>>> a[-1]
'!'
>>> a[-2]
'?'
\end{minted}
\end{pys}

Enfin on peut vouloir \textit{extraire une sous-chaîne} d'une chaîne : \mintinline{python}{chaine[p:q]} renvoie la sous-chaîne de caractère qui commence par
\mintinline{python}{chaine[p]} et qui se termine par \mintinline{python}{chaine[q-1]} :

\begin{pys}\begin{minted}{python}
>>> a = 'Salut'
>>> a[1:3]
'al'
\end{minted}
\end{pys}

Pour terminer, on peut vouloir tronquer à gauche ou à droite. La syntaxe est la même :\\
- \mintinline{python}{chaine[p:]} renvoie la sous-chaîne commençant par \mintinline{python}{chaine[p]}.\\
-  \mintinline{python}{chaine[:p]} renvoie la sous-chaîne se terminant par \mintinline{python}{chaine[p-1]}.

\begin{pys}\begin{minted}{python}
>>> a = 'Salut'
>>> a[:3] + '0' + a[3:]
'Sal0ut'
\end{minted}
\end{pys}

\section{Le type list}

Le type \mintinline{python}{list} sert à représenter des \textit{listes ordonnées}, indexées par $\lbrace 0;1;\ldots ;n\rbrace\quad (n\in\N)$.\\
Pour une variable \mintinline{python}{a} de type \mintinline{python}{list}, on accèdera à \mintinline{python}{a[0]}, \mintinline{python}{a[1]}\ldots\\
Les éléments d'une liste peut être de types différents.

\begin{pys}\begin{minted}{python}
>>> vide = list()
>>> vide
[]
>>> notes = [12, 11, 20, 9]
>>> notes[1]
11
>>> fouillis = [True, 1, 2.1, 'salut']
>>> fouillis[0]
True
\end{minted}
\end{pys}

On dispose des mêmes opérations que pour le type \mintinline{python}{str}: \mintinline{python}{+} sert à concaténer deux listes, et on peut multiplier une liste par un \mintinline{python}{int}.\\

On reviendra plus tard sur ce type très utile.

\section{Le type dict}
Le type \mintinline{python}{dict} pour représenter des \textit{dictionnaires}.\\
Les dictionnaires généralisent les listes car ils peuvent être indexés par des ensembles quelconques.\\
Pour une variable \mintinline{python}{couleur} de type \mintinline{python}{dict}, on aura par exemple \mintinline{python}{couleur['ciel']='bleu'}.

\begin{pys}\begin{minted}{python}
>>> chanteur = {'Nirvana': 'Kurt Cobain', 'U2': 'Bono'}
>>> chanteur['U2']
'Bono'
\end{minted}
\end{pys}


La notion de dictionnaire peut s'avérer assez compliquée à gérer (surtout au départ quand on manque de pratique) mais elle est très puissante et parfois très élégante.\\

\section{La fonction type et les conversions de type}

Pour connaître (ou vérifier) le type d'une variable, on peut utiliser la fonction \mintinline{python}{type}.

\begin{pys}\begin{minted}{python}
>>> a = 2
>>> type(a)
<class 'int'>
>>> type('hello')
<class 'str'>
>>> type(22 / 7)
<class 'float'>
\end{minted}
\end{pys}


Enfin, pour changer le type d'une variable, on peut utiliser les fonctions \mintinline{python}{int}, \mintinline{python}{float}, \mintinline{python}{str} et \mintinline{python}{list} (quand cela a un sens) comme des \textit{fonctions de conversion}.

\begin{pys}\begin{minted}{python}
>>> a = 2020
>>> b = float(a)
>>>	b
2020.0
>>> c = str(b)
>>> c
'2020.0'
\end{minted}
\end{pys}

\section{Et les constantes ?}


Certains langages permettent de définir des \textit{constantes} (\textsc{C++}, \textsc{Java} par exemple). Une constante est une variable dont la valeur est fixée lors de son initialisation et qui ne peut plus être modifiée ensuite.

\begin{encadrecolore}{Attention}{UGLiRed}
    On ne peut pas définir de constantes dans \textsc{Python}.\\
    Cependant on utilise la convention suivante : les variables écrites en majuscules sont à considérer comme des constantes.
\end{encadrecolore}

\begin{pyc}\begin{minted}{python}
TAILLE_MINI = 140
TAILLE_MAXI = 200
\end{minted}
\end{pyc}


\section{Quelques \og trucs \fg{} utiles}

\subsubsection*{Affectations multiples}
\textsc{Python} permet d'affecter plusieurs valeurs à plusieurs variables en même temps.

\begin{pys}\begin{minted}{python}
>>> a, b= 10, 2
>>> a
10
>>> b
2

>>> a, b = b, a
>>> a
2
>>> b
10
\end{minted}
\end{pys}

\subsubsection*{Notation condensée}
On est souvent amené à écrire des instructions telles que  \mintinline{python}{a = a + 1} ou \mintinline{python}{b = b / 2}. Cela peut être lourd quand les variables ne s'appellent
pas \mintinline{python}{a} ou \mintinline{python}{b} mais \mintinline{python}{rayon_sphere} ou \mintinline{python}{largeur_niveau}. On peut utiliser les notation suivantes :
\begin{pys}\begin{minted}{python}
>>> rayon_sphere = 3.4
>>> rayon_sphere /= 2
>>> rayon_sphere
1.7
>>> largeur_niveau = 19
>>> largeur_niveau += 1
>>> largeur_niveau
20
\end{minted}
\end{pys}

On dispose également de \mintinline{python}{*=}, \mintinline{python}{//=}, \mintinline{python}{\%=}, \mintinline{python}{-=} et \mintinline{python}{**=}.
\section{Exercices}

\begin{exercice}[ : fonction print]

    La fonction \mintinline{python}{print} sert à afficher à l'écran le contenu d'une variable, ou bien une chaîne de caractère.\\
    Recopier et observer :
\begin{minted}{python}
>>> print('Bonjour')
>>> a = 2020
>>> print('Nous sommes en ', a)
>>> b = 7
>>> print('Dans ', b,' ans, nous serons en ', a + b)
\end{minted}
\end{exercice}

\begin{exercice}[ : affectations 1]

    Qu'affiche le script suivant (ne pas chercher à l'écrire en machine) ?
\begin{minted}{python}
a = 1
b = a + 1
c = 2 * b - ( a + b - 4)
print(b, c)
\end{minted}
\end{exercice}

\begin{exercice}[ : affectations 2]
    Qu'affiche le script suivant ?
    
    \begin{minted}{python}
a, b = 2, 5
b, c = b + 1, a + b
print(c)
    \end{minted}
    \end{exercice}

    \begin{exercice}[ : affectations 3]
        Qu'affiche le script suivant ?
        \begin{minted}{python}    
a, b, c = 1, 2, 3
a, b, c = b, c, a
a, b, c = b, c, a
a, b, c = b, c, a
print(a, b, c)
        \end{minted}
    \end{exercice}

\begin{exercice}[ : fonction input ]

    C'est le pendant de la fonction \mintinline{python}{print} : elle permet à l'utilisateur d'entrer une chaîne de caractères à l'aide du clavier.

    Recopier et observer :
\begin{minted}{python}
>>> prenom = input('Quel est ton prénom ? ')
>>> reponse = 'Enchanté, ' + prenom
>>> print(reponse)
\end{minted}
\end{exercice}

\begin{exercice}

    Le script suivant produit une erreur :
\begin{minted}{python}
age = input('Quel est ton age ? ')
vieux = 10 + age
print('Dans 10 ans tu auras ', vieux, ' ans.')
\end{minted}
    Observer le message d'erreur (surtout la dernière ligne).\\
    À l'aide d'une \textit{conversion de type}, rectifier le script.
\end{exercice}
\begin{exercice}
D'où vient le problème ? Proposer une solution (deux si tu es astucieu\cdot x\cdot se).
\begin{minted}{python}
>>> chaine = 'Mon nombre préféré est le '
>>> nb = 7
>>> print(chaine+nb)
\end{minted}
\end{exercice}

\begin{exercice}
    Donner 3 manières de définir en une ligne une variable \mintinline{python}{a} de type \mintinline{python}{float} valant 2.
\end{exercice}

\begin{exercice}
    Que fait le script suivant (essayer de le comprendre sans le taper) ?
\begin{minted}{python}
pouces = input('Entrez la valeur en pouces à convertir : ')
pouces = float(pouces)
print('Cela fait ', pouces * 2.54, ' centimètres.')
\end{minted}

    \'Ecrire un script qui demande la taille d'un fichier en kilooctets (ko) puis la convertit en octets (1ko = $2^{10}$ octets).
\end{exercice}
\begin{exercice}
    Le numéro de sécurité sociale est constitué de 13 chiffres auquel s'ajoute la clé de contrôle (2 chiffres).\\
    Exemple : $$\underbrace{1\ 89\ 11\ 26\ 108\ 268}_{\textrm{chiffres}}\ \underbrace{91}_{\textrm{clé}}$$
    La clé de contrôle est calculée par la formule : 97 - (numéro de sécurité sociale modulo 97).\\
    À l'écrit, retrouver la clé de contrôle de votre numéro de sécurité sociale (ou à défaut, de l'exemple).\\
    Quel est l'intérêt de la clé de contrôle ?\\
    \'Ecrire un script qui, à partir des 13 chiffres du numéro de sécurité sociale, affiche le numéro complet.
\end{exercice}
\begin{exercice}
    L'identifiant d'accès au réseau du lycée est construit de la manière suivante : initiale du prénom puis les 8 premiers caractères du nom (le tout en minuscule).\\
    Exemple : Alexandre Lecouturier donne \texttt{alecoutur}.\\
    \'Ecrire un script qui, à partir des deux variables \texttt{prenom} et \texttt{nom}, affiche l'identifiant.\\
    Tu pourras créer une variable \mintinline{python}{nom}, une variable \mintinline{python}{prenom} et utiliser la fonction \mintinline{python}{input}.\\
    Pour passer \mintinline{python}{nom} et \mintinline{python}{prenom} en minuscule, il suffit de taper :
    \begin{minted}{python}
nom = nom.lower()
prenom = prenom.lower()
\end{minted}
\end{exercice}

%!TEX root = cours.tex
\chapter{Tests et conditions}
\introduction{Ceci n'est pas un test !}
\section{Des outils pour comparer}

Ce sont les \textit{opérateurs de comparaison} :\\

{\small
\alternaterowcolors
\begin{tabular}{ccc}
	\rowcolor{lightgray}
	\rowcolor{UGLiOrange}{\boxfont\color{white} Opérateur} & {\boxfont\color{white} Signification} & {\boxfont\color{white} Remarques}                                                                     \\

	\pythoninline{<}                                       & strictement inférieur                 & Ordre usuel sur \pythoninline{int} et \pythoninline{float}, lexicographique sur \pythoninline{str}... \\

	\pythoninline{<=}                                      & inférieur ou égal                     & Idem                                                                                                  \\

	\pythoninline{>}                                       & strictement supérieur                 & Idem                                                                                                  \\

	\pythoninline{>=}                                      & supérieur ou égal                     & Idem                                                                                                  \\

	\pythoninline{==}                                      & égal                                  & \og avoir même valeur\fg\  \textit{Attention :} deux signes =                                         \\

	\pythoninline{!=}                                      & différent                             &                                                                                                       \\

	\pythoninline{is}                                      & identique                             & être le même objet                                                                                    \\

	\pythoninline{is not}                                  & non identique                         &                                                                                                       \\

	\pythoninline{in}                                      & appartient à                          & avec \pythoninline{str}, \pythoninline{list} et \pythoninline{dict}                                   \\

	\pythoninline{not in}                                  & n'appartient pas à                    & avec \pythoninline{str} et \pythoninline{list} et \pythoninline{dict}                                 \\
\end{tabular}
}\normalsize

\begin{pys}
	\begin{minted}{python}
>>> a = 2
>>> a == 2
>>> a == 3
>>> a == 2.0
>>> a is 2.0
>>> a != 100
>>> a > 2
>>> a >= 2
    \end{minted}
\end{pys}

\begin{pys}
	\begin{minted}{python}
>>> a = 'Alice'
>>> b = 'Bob'
>>> a < b
>>> 'e' in a
>>> 'e' in b
>>> liste = [1,10,100]
>>> 2 in liste
    \end{minted}
\end{pys}

Ces opérateurs permettent de réaliser des tests basiques. Pour des tests plus évolués on utilisera des \og mots de liaison \fg{} logiques.

\section{Les connecteurs logiques}

\begin{itemize}
	\item   \pythoninline{and} permet de vérifier que 2 conditions sont \textit{vérifiées simultanément}.
	\item   \pythoninline{or} permet de vérifier qu'\textit{au moins une} des deux conditions est vérifiée.
	\item   \pythoninline{not} est un opérateur de \textit{négation} très utile quand on veut par exemple vérifier qu'une condition est fausse.
\end{itemize}
Voici les tables de vérité des deux premiers connecteurs :

\begin{center}
	\begin{tabular}{|c|c|c|}
		\hline
		\pythoninline{and} & \textit{True}        & \textit{False}       \\
		\hline
		\textit{True}      & \pythoninline{True}  & \pythoninline{False} \\
		\hline
		\textit{False}     & \pythoninline{False} & \pythoninline{False} \\
		\hline
	\end{tabular}\hspace{2em}
	\begin{tabular}{|c|c|c|}
		\hline
		\pythoninline{or} & \textit{True}       & \textit{False}       \\
		\hline
		\textit{True}     & \pythoninline{True} & \pythoninline{True}  \\
		\hline
		\textit{False}    & \pythoninline{True} & \pythoninline{False} \\
		\hline
	\end{tabular}
\end{center}
À ceci on peut ajouter que \pythoninline{not True} vaut \pythoninline{False} et vice-versa.

\begin{pys}
	\begin{minted}{python}
>>> True and False
>>> True or False
>>> not True
    \end{minted}
\end{pys}

\begin{pys}
	\begin{minted}{python}
>>> resultats = 12.8
>>> mention_bien = resultats >= 14 and resultats < 16
>>> print(mention_bien)
    \end{minted}
\end{pys}

\section{if, else et elif}
Voici le schéma de fonctionnement d'un test \pythoninline{if} :
\begin{center}
	\includegraphics[height=10cm]{img/if}
\end{center}

\textbf{Attention :} Un bloc conditionnel doit être \textit{tabulé} par rapport à la ligne précédente : il n'y a ni \pythoninline{DébutSi}  ni \pythoninline{FinSi}
en \textsc{Python}, ce sont les tabulations qui délimitent les blocs.

\begin{pyc}
	\begin{minted}{python}
phrase='Je vous trouve très joli'
reponse = input('Etes vous une femme ?(O/N) : ')
if reponse == 'O':
    phrase += 'e'
phrase +='.'
print(phrase)
\end{minted}
\end{pyc}

Voici le schéma de fonctionnement d'un test \pythoninline{if...else} :
\begin{center}
	\includegraphics[height=10cm]{img/ifelse}
\end{center}

\begin{pyc}
	\begin{minted}{python}
print('Bonjour')
age = int(input('Entrez votre age : '))
if age >= 18:
    print('Vous etes majeur')
else:
    print('Vous etes mineur.')
print('Au revoir.')
\end{minted}
\end{pyc}

Voici un exemple de fonctionnement d'un test \pythoninline{if...elif...} :
\begin{pyc}
	\begin{minted}{python}
print('Bonjour')
prenom = input('Entrez un prénom : ')
if prenom == 'Robert':
    print("Robert, c'est le prénom de mon grand-père.")
elif prenom == 'Raoul':
    print("Mon oncle s'appelle Raoul.")
elif prenom == 'Médor':
    print("Médor, comme mon chien !")
else:
    print("Connais pas")
print('Au revoir.')
\end{minted}
\end{pyc}
Et voici un schéma décrivant son fonctionnement :
\begin{center}
	\includegraphics[width=\linewidth]{img/ifelifelse}
\end{center}



On peut bien sûr inclure autant de \pythoninline{elif} que nécessaire.

\section{Exercices}

%----------------------------------------------------------------------
\begin{exercice}
	\'Ecrire un script qui demande son âge à l'utilisateur puis qui affiche \pythoninline{'Bravo pour votre longévité.'} si celui-ci est supérieur à 90.
\end{exercice}

\begin{exercice}[]
	\'Ecrire un script qui demande un nombre à l'utilisateur puis affiche si ce nombre est pair ou impair.
\end{exercice}
%----------------------------------------------------------------------
\begin{exercice}
	\'Ecrire un script qui demande l'âge d'un enfant à l'utilisateur puis qui l'informe ensuite de sa catégorie :
	\begin{itemize}
		\item   trop petit avant 6 ans;
		\item   poussin de 6 à 7 ans inclus;
		\item   pupille de 8 à 9 ans inclus;
		\item   minime de 10 à 11 ans inclus;
		\item   cadet à 12 ans et plus;
	\end{itemize}
\end{exercice}
%----------------------------------------------------------------------

\begin{exercice}
	\'Ecrire un script qui demande une note sur 20 à l'utilisateur puis vérifie qu'elle est bien comprise entre 0 et 20. Si c'est le cas rien ne se produit mais sinon le programme devra afficher un message tel que \pythoninline{'Note non valide.'}.
\end{exercice}

\begin{exercice}
	\'Ecrire un script qui demande un nombre à l'utilisateur puis affiche s'il est divisible par 5, par 7 par aucun ou par les deux de ces deux nombres.
\end{exercice}
%----------------------------------------------------------------------
\begin{exercice}
	En reprenant l'exercice du chapitre 1 sur les numéros de sécurité sociale, écrire un script qui demande à un utilisateur son numéro de sécurité sociale, puis qui vérifie si la clé est valide ou non.
\end{exercice}


%----------------------------------------------------------------------
\begin{exercice}
	\'Ecrire un script qui résout dans $\R$ l'équation du second degré $ax^2+bx+c=0$.\\
	On commencera par \pythoninline{from math import sqrt} pour utiliser la fonction \pythoninline{sqrt}, qui calcule la racine carrée d'un \pythoninline{float}.

	On rappelle que lorsqu'on considère une équation du type $ax^2+bx+c=0$
	\begin{itemize}
		\item   si $a=0$ ce n'est pas une équation de seconde degré;
		\item   sinon on calcule $\Delta = b^2-4ac$ et
		      \begin{itemize}
			      \item   Si $\Delta<0$ l'équation n'a pas de solutions dans $\R$;
			      \item   Si $\Delta=0$ l'équation admet pour unique solution $\dfrac{-b}{2a}$;
			      \item   Si $\Delta>0$ l'équation admet 2 solutions : $\dfrac{-b-\sqrt{\Delta}}{2a}$ et $\dfrac{-b+\sqrt{\Delta}}{2a}$.
		      \end{itemize}
	\end{itemize}
	Pour vérifier que le script fonctionne bien on pourra tester les équations suivantes :
	\begin{itemize}
		\item   $2x^2+x+7=0$ (pas de solution dans $\R$);
		\item   $9x^2-6x+1=0$ (une seule solution qui est $\dfrac{1}{3}$);
		\item   $x^2-3x+2$ (deux solutions qui sont 1 et 2).
	\end{itemize}
\end{exercice}

%----------------------------------------------------------------------
\begin{exercice}
	L'opérateur \pythoninline{nand} est défini de la manière suivante : si \pythoninline{A} et \pythoninline{B} sont deux booléens alors
	\begin{center}
		\pythoninline{A nand B} vaut \pythoninline{not (A and B)}
	\end{center}
	Construire la table de vérité de \pythoninline{nand} en complétant :
	\begin{center}

		\begin{tabular}{|c|c|c|c|}
			\hline
			\rowcolor{UGLiOrange}{\boxfont\color{white} A} & {\boxfont\color{white} B} & {\boxfont\color{white} A and B} & {\boxfont\color{white} not (A and B)} \\
			\hline
			\pythoninline{False}                           & \pythoninline{False}      &                                 &                                       \\
			\hline
			\pythoninline{False}                           & \pythoninline{True}       &                                 &                                       \\
			\hline
			\pythoninline{True}                            & \pythoninline{False}      &                                 &                                       \\
			\hline
			\pythoninline{True}                            & \pythoninline{True}       &                                 &                                       \\
			\hline
		\end{tabular}
	\end{center}
\end{exercice}



%!TEX root = <cours.tex>
\chapter{Boucles}
\introduction{Tant que tu n'y arrives pas recommence.}
On s'intéresse dans ce chapitre aux \textit{structures itératives}, plus communément appelée \textit{boucles}.\\


\section{La boucle while}

Voici son schéma de fonctionnement :
\begin{center}
    \includegraphics[scale=0.2]{img/while.png}
\end{center}

La boucle \pythoninline{while} exécute un bloc d'instructions conditionnel \textit{tant que} une condition est vérifiée.\\
Dès que la condition n'est plus vérifiée, le bloc conditionnel n'est plus exécuté.

\begin{pys}
\begin{minted}{python}
reponse=''
print('Bonjour !')
while reponse !='n':
    reponse = input('Voulez-vous continuer ? (o/n) : ')
print('Au revoir.')
\end{minted}
\end{pys}
Les boucles \pythoninline{while} doivent être utilisées avec soin : si la condition est toujours vérifiée, le programme ne s'arrêtera pas :

\begin{pyc}
\begin{minted}{python}
while True:
    print('Au secours !')
\end{minted}
\end{pyc}
Voici un exemple typique d'utilisation de la boucle \pythoninline{while} : \\

On place un capital de 2000 euros sur un compte à intérêts annuels de 2\%. On aimerait savoir au bout de combien de temps, sans rien toucher, le 
solde du compte dépassera 2300 euros.\\

\begin{pyc}
\begin{minted}{python}
 solde = 2000  # solde initial
n = 0  # nombre d'annees
while solde <= 2300:  # condition de boucle
    n += 1  # augmente le compteur d'annees
    solde *= 1.02  # actualise le solde
print(' Il nous faudra ', n, 'ans.')  # affichage final   
\end{minted}
\end{pyc}

\section{La boucle for ... in range(...)}

Commençons examiner un nouveau type : \pythoninline{range} (plage de valeurs)

\begin{pys}
\begin{minted}{python}
>>> a = range(10)
>>> type(a)
>>> print(list(a))
\end{minted}
\end{pys}

Si \pythoninline{range(10)} ressemble beaucoup à la liste \pythoninline{[0,1,...,9]}, la finalité de \\\pythoninline{range(10)} est d'être un \textit{itérateur}, 
c'est-à-dire une objet dont on peut parcourir le contenu pour créer une boucle :

\begin{pyc}
\begin{minted}{python}
for i in range(10):
    print(i)
\end{minted}
\end{pyc}
La syntaxe complète de \pythoninline{range} est : \pythoninline{range(<debut>,fin,<increment>)}.\\

Par défaut, si ce n'est pas précisé, \pythoninline{debut=0}, et \pythoninline{increment=1}.\\

\pythoninline{range(<debut>,fin,<increment>)} renvoie la plage de valeurs suivantes :
\begin{itemize}
    \item   On part de la valeur de début, appelons la \pythoninline{val}
    \item   Tant que \pythoninline{val < fin}:\\
    $\qquad$ - ajouter \pythoninline{val} à la plage\\
    $\qquad$ - ajouter \pythoninline{increment} à \pythoninline{val}
\end{itemize}
Ainsi, \pythoninline{range(2,52,10}) renvoie la plage de valeurs \pythoninline{2,12,22,32,42}, mais\\ \pythoninline{range(2,53,10}) renvoie la plage de valeurs 
\pythoninline{2,12,22,32,42,52}.\\

Très souvent, on se contente d'utiliser une instruction du type \pythoninline{range(n)}, où \pythoninline{n} est de type \pythoninline{int}.\\


Voici un exemple : Calculons $1+2+\ldots+100$ :

\begin{pyc}
\begin{minted}{python}
somme = 0
for i in range(1, 101):
    somme += i
print(somme)
\end{minted}
\end{pyc}

\section{La boucle for ... in ...}

On peut généraliser le paragraphe précédent à toute \textit{variable itérable}, c'est extrêmement puissant : les \pythoninline{str}, les \pythoninline{list} et les 
\pythoninline{dict} sont des types itérables.\\

Voici des exemples :\\

Comptons le nombre de voyelles d'une chaîne de caractères :

\begin{pyc}
\begin{minted}{python}
voyelles = 'aeiouy'  # ensemble de voyelles
phrase = input('Entrez une phrases sans accents : ').lower()  # phrase mise en minuscules
compteur = 0  # comptera les voyelles
for lettre in phrase:  # on parcourt la phrase
    if lettre in voyelles:  # est-ce une voyelle ?
        compteur += 1  # si oui on comptabilise
print('Nombre de voyelles : ', compteur)  # affiche le nombre
\end{minted}
\end{pyc}

Faisons la moyenne d'une liste de notes :


\begin{pyc}
\begin{minted}{python}
liste_notes = [12, 11.5, 13, 18, 13, 11, 9]
moyenne = 0
for note in liste_notes:
    moyenne += note
moyenne /= len(liste_notes)
print(moyenne)
\end{minted}
\end{pyc}

Pour le dernier exemple on utilise le type \pythoninline{dict} : soit \pythoninline{a} une variable de ce type :
\begin{itemize}
    \item   \pythoninline{a.keys()} renvoie la liste des clés (des indices du dictionnaire).
    \item   \pythoninline{a.values()} renvoie la liste des valeurs prises par le dictionnaire.
\end{itemize}

Voici un second programme de moyenne :

\begin{pyc}
\begin{minted}{python}
resultats = {'EPS': 12, 'maths': 15, 'info': 18}
moyenne = 0
for note in resultats.values():
    moyenne += note
moyenne /= len(resultats)
print(moyenne)
\end{minted}
\end{pyc}

\section{Quelle boucle utiliser ?}

Si la boucle dépend d'une condition particulière on préfèrera la boucle \pythoninline{while}.

Si le nombre d'itérations de la boucle est connu on préfèrera une boucle \pythoninline{for}.

On peut utiliser une boucle \pythoninline{for} sur toute \textit{structure itérative}, par exemple 
    une variable de type \pythoninline{range}, \pythoninline{str}, \pythoninline{list} ou, dans une certaine mesure, \pythoninline{dict}.
    
\section{Exercices}
\begin{exercice}
    Calculer à l'aide d'un script la somme des carrés des 1000 premiers entiers non nuls.
\end{exercice}

%----------------------------------------------------------------------
\begin{exercice}
    Calculer à l'aide d'un script la somme des carrés des 1000 premiers multiples de 3 non nuls.
\end{exercice}

%---------------------------------------------------------------------
\begin{exercice}
    \'Ecrire un script qui demande une phrase à l'utilisateur, puis affiche la phrase en rajoutant des tirets.\\
    Exemple : on entre \pythoninline{'Salut à toi'} le script affiche \pythoninline{'S-a-l-u-t- -à- -t-o-i-}.
\end{exercice}

\begin{exercice}
    Calculer à l'aide d'un script le nombre $n$ à partir duquel la somme $1^2+2^2+\ldots+n^2$ dépasse un milliard.
\end{exercice}

%----------------------------------------------------------------------
\begin{exercice}
    \'Ecrire un script qui demande une phrase et compte le nombre d'occurrences de la lettre \og a \fg{} dans celle-ci.
\end{exercice}

%----------------------------------------------------------------------
\begin{exercice}
    Programmer le jeu du "plus petit plus grand" :
    \begin{itemize}
        \item   L'ordinateur choisit un nombre entier au hasard compris entre 0 et 100.\\
        Au début du script, importer la fonction \pythoninline{randint} du module \pythoninline{random} avec \pythoninline{form random import randint}.\\
        Pour obtenir un entier au hasard, utiliser \pythoninline{randint(0,100)}.
        \item   L'utilisateur propose un nombre, l'ordinateur répond \og gagné\fg, \og plus petit\fg{} ou \og plus grand\fg.
        \item   Le programme continue tant que l'utilisateur n'a pas gagné.
    \end{itemize}
\end{exercice}

%----------------------------------------------------------------------
\begin{exercice}
    On considère la suite $s$ définie par:
    $$\left\{
    \begin{array}{llll}
    s_0 & = & 1000 & \\
    s_{n+1} & = & 0,99s_n+1 & \textrm{pour tout } n\in\mathbf{N}
    \end{array}
    \right. $$
    
    \begin{itemize}
        \item   \'Ecrire un script calculant les premiers termes de $s$ (vous décidez le nombre de termes).
        \item   Utiliser ce script pour conjecturer la limite de $s$.
        \item   Modifier ce script pour obtenir le plus petit entier $n$ tel que l'écart entre $s_n$ et sa limite soit inférieur ou égal à $10^{-4}$.
        
    \end{itemize}
\end{exercice}

%----------------------------------------------------------------------
\begin{exercice}
    $\varphi$ (lettre phi, équivalent du \og f \fg{} en grec) est défini par : $$\varphi=1+\frac{1}{1+\frac{1}{1+\frac{1}{\ldots}}}$$
    Sur papier, fabriquer une suite par récurrence commençant ainsi :
    \begin{itemize}
        \item   1
        \item   $1+\frac{1}{1}$
        \item   $1+\frac{1}{1+\frac{1}{1}}$
        \item   Et c\ae tera (trouver une relation simple pour calculer le terme suivant à partir du terme actuel).
    \end{itemize}
    Programmer un script qui calcule successivement les termes de cette suite (aller jusqu'à 10000\eme).\\
    
    Comparer avec la valeur exacte de $\varphi$, qui est $\frac{1+\sqrt{5}}{2}$.
\end{exercice}

\begin{exercice}
    \'Ecrire un script qui détermine si un entier est premier ou pas.
\end{exercice}
\chapter{Fonctions}
\introduction{Quelle est la fonction de ce chapitre ?}
\section{Exemples de fonctions}
\subsection{Un objet déjà connu}
Nous avons déjà rencontré des fonctions \textit{côté utilisateur}:
\begin{itemize}
    \item \mintinline{python}{input}
          \begin{itemize}
              \item prend en entrée une chaîne de caractères ;
              \item renvoie la chaîne de caractère saisie par l'utilisateur.
          \end{itemize}
          On peut noter ceci \mintinline{python}{input(chaine: str) -> str}
    \item \mintinline{python}{len}
          \begin{itemize}
              \item prend en entrée une liste ;
              \item renvoie le nombre d'éléments de cette liste.
          \end{itemize}
          On peut noter cela \mintinline{python}{len(lst: list) -> int}
\end{itemize}

\subsection{De multiples formes}
\floatpictureright{0.5}{ch-fonctions/img/fonction1}
{Les deux exemples précédents rentrent dans la catégorie représentée à droite.}
\medskip\par
\floatpictureleft{0.5}{ch-fonctions/img/fonction2}
{Certaines fonctions sont comme à gauche.\\
    Par exemple \mintinline{python}{max(20,3,10)} renvoie 20.}
\medskip\par
\floatpictureright{0.5}{ch-fonctions/img/fonction3}{
    D'autres fonctions sont comme à droite.\\
    On verra des exemples plus tard.}
\medskip\par
\floatpictureleft{0.5}{ch-fonctions/img/fonction4}{
    D'autres encore sont comme à gauche.\\
    Par exemple \mintinline{python}{print("salut")} ne renvoie rien mais affiche \mintinline{python}{salut} à l'écran.}
\medskip\par
\floatpictureright{0.5}{ch-fonctions/img/fonction5}{
    D'autres suivent le schéma ci-contre.\\
    Par exemple dans le module \mintinline{python}{time}, la fonction \mintinline{python}{time} ne prend aucun paramètre d'entrée mais renvoie l'heure qu'indique l'horloge de l'ordinateur.\\
    On peut par exemple l'utiliser pour stocker une heure précise en tapant \mintinline{python}{maintenant = time()}.}
\medskip\par
\floatpictureleft{0.5}{ch-fonctions/img/fonction6}{
    Enfin certaines suivent ce schéma.\\
    Par exemple dans le module \mintinline{python}{pygame}, \mintinline{python}{pygame.display.flip} ne prend aucun paramètre d'entrée, ne renvoie aucune valeur, mais actualise la fenêtre graphique.\\
    On l'appelle donc en tapant \mintinline{python}{pygame.display.flip()}.}
\medskip\par
Il est possible de créer de nouvelles fonctions.\\
On parle alors de fonctions \textit{côté concepteur}.\\

Il faut donc définir rigoureusement ce qu'est une fonction.

\section{Définition de la notion de fonction}
\begin{definition}[ : fonction]
    Une \textit{fonction} est un \og morceau de code\fg{} qui représente un \textit{sous-programme}.\\
    Elle a pour but d'effectuer une tâche \textit{de manière indépendante}.\\
\end{definition}

\begin{exemple}
    On veut modéliser la fonction mathématique $f$ définie pour tout nombre réel $x$ par $$f(x)=x^2+3x +2$$

    On écrira alors

    \begin{minted}[breaklines,breakanywhere]{python}
def f(x : float) -> float:
    return x ** 2 + 3 * x + 2
            \end{minted}

    Pour évaluer ce que vaut $f(10)$ et affecter cette valeur à une variable, on pourra désormais écrire \mintinline{python}{resultat = f(10)}.
\end{exemple}


Que fait la fonction \mintinline{python}{mystere} ?

\begin{minted}[breaklines,breakanywhere]{python}
def mystere(a : float, b : float) -> float:
    if a <= b:
        return b
    else:
        return a            
\end{minted}



La fonction \mintinline{python}{mystere}:
\begin{itemize}
    \item   prend en entrée deux paramètres de type \mintinline{python}{float} \mintinline{python}{a} et \mintinline{python}{b};
    \item   renvoie le plus grand de ces deux nombres.
\end{itemize}

La réponse que l'on vient de formuler s'appelle \textit{la spécification} de la fonction $f$.

\begin{definition}[ : fonction]
    Donner la spécification d'une fonction \mintinline{python}{f} c'est
    \begin{itemize}
        \item   préciser le(s) type(s) du (des) paramètre(s) d'entrée (s'il y en a) ;
        \item   indiquer sommairement ce que fait la fonction \mintinline{python}{f} ;
        \item   préciser le(s) type(s) de la (des) valeur(s) de sortie (s'il y en a).
    \end{itemize}
\end{definition}

\section{Anatomie d'une fonction}
\begin{pyc}
\begin{minted}{python}
def f(lst: list) -> int
    mini = lst[0]
    n = len(lst)
    for i in range(n):
        lst[i] < mini:
        mini = lst[i]
    return mini
   \end{minted}
\end{pyc}

La fonction \mintinline{python}{f}
\begin{itemize}
    \item   prend en entrée une liste (sous entendu d'entiers);
    \item   renvoie le plus petit entier de cette liste.
\end{itemize}

\subsection{Paramètre formel}

\floatpictureleft{0.5}{ch-fonctions/img/anat1}{
Le paramètre d'entrée est \textit{formel} : \textit{le nom de cette variable n'existe qu'à l'intérieur de la fonction}.
Si ce nom de variable existe déjà à l'extérieur de la fonction, \textit{ce n'est pas la même variable}.    
}\medskip\par

\floatpictureright{0.5}{ch-fonctions/img/anat4}{
Le type du paramètre d'entrée peut être spécifié. Ce n'est pas obligatoire mais très fortement recommandé pour \og garder les idées claires\fg{}.
}\medskip\par

\subsection{Variables locales}
\floatpictureleft{0.5}{ch-fonctions/img/anat2}{
Toutes les variables \textit{créées} dans une fonction n'existent \textit{que dans cette fonction}. Elles ne sont pas accessibles depuis l'extérieur de la fonction. On dit que ce sont des \textit{variables locales}.}\medskip\par

\subsection{valeur de sortie}
\floatpictureright{0.5}{ch-fonctions/img/anat3}{
Le type de la valeur de sortie peut être précisé, c'est également recommandé.    
}

\section{En pratique}
\subsection{Des exemples}
 \begin{pyc}
\begin{minted}{python}
1   def f(x : float) -> float:
2       return x ** 2 + 3 * x + 2
3    
4   print(f(1)) # Affiche 6
\end{minted}
\end{pyc}
Le programme commence à la ligne 4 !\\
Les 2 premières lignes servent à définir la fonction \texttt{f}, elles ne sont exécutées que lorsqu'on évalue \texttt{f(1)}.\\



\begin{pyc}
\begin{minted}{python}
def f(x : float) -> float:
    return x ** 2 + 3 * x + 2
    
print(x) # Provoque une erreur
\end{minted}
\end{pyc}
L'erreur vient du fait que la variable \texttt{x} \textit{n'est pas définie}. Le \og \texttt{x} qu'on voit dans la fonction \texttt{f}\fg{} est un paramètre formel et n'existe que dans \texttt{f}.\\


\begin{pyc}
\begin{minted}{python}
def f(x : float) -> float:
    a = 2
    return x + a

print(a) # Provoque une erreur
    \end{minted}
\end{pyc}
L'erreur vient du fait que la variable \texttt{a} \textit{est locale} : elle n'est définie que durant l'exécution de \texttt{f}.\\

\begin{pyc}
\begin{minted}{python}
def f(x : float) -> float:
    a = 2
    return x + a

print(f(4)) # Affiche 6        
print(a) # Provoque une erreur
    \end{minted}
\end{pyc}
C'est encore la même erreur : une fois \texttt{f(4)} évaluée, \texttt{a} n'existe plus.\\

\begin{pyc}
\begin{minted}{python}
1   def f(x : float) -> float:
2       a = 2
3       return x + a
4
5   a = 3
6   print(f(4)) # Affiche 6
7   print(a) # Affiche 3 et pas 2
\end{minted}
\end{pyc}

La variable \texttt{a} définie dans la fonction \texttt{f} n'est pas la même que celle qui est définie à la ligne 5.\\
Celle définie à la ligne 2 est \textit{locale}.\\
La variable \texttt{a} de la ligne 5 est appelée \textit{globale}.\\



\begin{pyc}
\begin{minted}{python}
def f(x : float) -> float:
    return x + a
        
a = 3
print(f(4)) # Affiche 7
\end{minted}
\end{pyc}
\begin{aretenir}
Une fonction a le droit d'\textit{accéder en lecture} à une variable globale, mais n'a pas \textit{a priori} le droit d'en modifier la valeur.
\end{aretenir}

\subsection{À éviter autant que possible}



\begin{pyc}
\begin{minted}{python}
1   def f(x : float) -> float:
2       global a 
3       a = a + 1
4       return x + a
5        
6   a = 3
7   print(f(4)) # Affiche 8
8   print(a) # Affiche 4       
\end{minted}
\end{pyc}
À la ligne 2, on signale à Python que \texttt{f} a la droit de modifier la variable globale \texttt{a}.
C'est fortement déconseillé : sauf si on ne peut pas faire autrement, une fonction ne doit pas modifier les variables globales.
\part{Algorithmique}
\chapter{Recherche dichotomique}
\introduction{Plus petit ou plus grand ?}

\section{Présentation de l'algorithme}
Lorsqu'on dispose d'une liste d'éléments et d'un élément particulier, on peut rechercher si cet élément appartient ou non à la liste : il suffit de parcourir les éléments de la liste un par un et de les comparer à l'élément que l'on cherche. Cette démarche peut être améliorée si la liste possède des propriétés particulières, notamment si c'est une liste d'entiers triée.\\

On veut écrire une fonction \texttt{recherche\_dichotomique} qui :
\begin{itemize}
    \item   En entrée prend \begin{itemize}
                                \item   une liste \pythoninline{liste_triee} de $n$ entiers \textit{triée dans l'ordre croissant};
                                \item   un entier \pythoninline{val}.
                            \end{itemize}
    \item  Renvoie \begin{itemize}
                        \item   l'indice de \pythoninline{val} dans \pythoninline{liste_triee} si \pythoninline{val} appartient à \pythoninline{liste_triee};
                        \item   -1 si a n'appartient pas à \pythoninline{liste_triee}.
                    \end{itemize} 
\end{itemize}

\begin{exemple}[]
\begin{itemize}
    \item   \pythoninline{recherche_dichotomique([11, 20, 32, 33, 54], 32)} \\renvoie 2 car 32 est l'élément d'indice 2 de la liste.
    \item   \pythoninline{recherche_dichotomique([20, 32, 33, 54], 40)} \\renvoie -1 car 40 ne figure pas dans la liste.
\end{itemize}
\end{exemple}

\begin{methode}[]
On compare \pythoninline{val} avec l'élément $m$ qui se situe \og à peu près au milieu de \pythoninline{liste_triee}\fg{}.
\begin{itemize}
    \item   si $val$ est égal à $m$ c'est gagné, on renvoie l'indice de $m$ dans  \pythoninline{liste_triee};
    \item   sinon si $val>m$ on recommence avec la liste des éléments situés après $m$.
    \item   sinon c'est que $val<m$ et on recommence avec la liste des éléments situés avant $m$.
\end{itemize}
On itère ce procédé tant que la liste des valeurs à examiner n'est pas vide. Si on arrive à une liste vide c'est que \pythoninline{val} , n'est pas dans \pythoninline{liste_triee}
\end{methode}
Voici l'algorithme traduit en \textsc{Python} :
\begin{pys}
\begin{minted}{python}
def recherche_dichotomique(liste_triee, val):
    gauche = 0  # début de la plage de valeurs à regarder
    droite = len(liste_triee) - 1  # fin de la plage
    while gauche <= droite:  # tant que la plage est non vide
        milieu = (gauche + droite) // 2  # on prend grosso modo le milieu
        if liste_triee[milieu] == val:
            return milieu  # si on trouve val au milieu c'est gagné
        elif liste_triee[milieu] > val:  # si on a dépassé val
            droite = milieu - 1  # alors on regarde avant
        else:
            gauche = milieu + 1  # sinon on regarde après
    # si on est sorti de la boucle
    return -1  # c'est qu'on n'a pas trouvé val
\end{minted}
\end{pys}

\section{\'Etude de l'algorithme}

Trois question se posent :

\begin{enumerate}
    \item   Pourquoi la boucle \textit{tant que} s'arrête-t-elle toujours ?\\ On dit que c'est un problème de \textit{terminaison}.
    \item   Quand la fonction renvoie -1, est-ce que cela veut bien dire que \pythoninline{val} n'est pas dans \pythoninline{liste_triee} ? De même quand la fonction renvoie une valeur $m\neq -1$, est-ce que cela veut bien dire que \pythoninline{val==liste_triee[m]} ?.\\
    C'est un problème de \textit{correction}.
    \item   Pourquoi cette fonction est-elle plus rapide qu'un parcours des éléments un par un ?\\ C'est un problème de \textit{complexité}
\end{enumerate}

\section{Terminaison}

Pour prouver qu'une boucle \textit{tant que} se termine, on détermine un \textit{variant} de boucle.

\begin{definition}[]
Un variant de boucle est un \textit{entier positif qui décroît strictement à chaque itération de boucle}. On le choisit de sorte à ce que lorsqu'il atteint zéro (ou un, en tout cas une petite valeur) la boucle se termine.\\
\end{definition}

Dans notre cas, le variant de boucle est l'entier \pythoninline{v} défini par \pythoninline{v = droite-gauche} :
\begin{itemize}
    \item   la condition du \pythoninline{while} est liée à \pythoninline{v} puisqu'elle se réécrit \pythoninline{while v >= 0};
    \item   quand on rentre dans la boucle on définit \pythoninline{milieu} et on a toujours \\
    \pythoninline{gauche <= milieu <= droite};
    \item   si \pythoninline{ liste_triee[milieu] == val} alors la boucle s'arrête.
        
    \item   sinon si \pythoninline{ liste_triee[milieu] > val} alors :\\
            \pythoninline{droite = milieu - 1} et donc\\
            \pythoninline{v} devient (appelons-le \texttt{v2} temporairement) \pythoninline{v2 = milieu - 1 - gauche} donc\\
            \pythoninline{v2 < milieu - gauche} et puisque \pythoninline{milieu <= droite} on a\\
            \pythoninline{v2 < droite - gauche} et ainsi \pythoninline{v2 < v}.
    \item   sinon \pythoninline{ liste_triee[milieu] < val} et à ce moment :\\
            \pythoninline{gauche = milieu + 1} et donc\\
            \pythoninline{v} devient (appelons-le \texttt{v2} temporairement) \pythoninline{v2 = droite - (milieu+1)} donc\\
            \pythoninline{v2 < droite - milieu} et puisque \pythoninline{gauche <= milieu } on a\\
            \pythoninline{v2 < droite - gauche} et ainsi \pythoninline{v2 < v}.         
\end{itemize}

Ainsi les valeurs de v décroissent strictement, donc finissent (si on ne trouve pas \texttt{val}) par atteindre zéro et la boucle se termine.\\
On dit qu'on a prouvé la \textit{terminaison} de la fonction.

\section{Correction}

Pour prouver que cette fonction est correcte, on doit utiliser un \textit{invariant de boucle} (ne pas confondre avec le \textit{variant de boucle}).

\begin{definition}[]
Un \textit{invariant de boucle} est une propriété P dépendant éventuellement des variables du programme
\begin{itemize}
    \item   P doit être vraie avant l'entrée dans la boucle \textit{tant que};
    \item   P doit rester vraie à chaque itération de boucle;
    \item   à la fin de la boucle, P doit nous permettre de conclure que la fonction \og fait bien ce qu'elle doit faire\fg{}.
\end{itemize}
\end{definition}


Dans notre cas voici l'invariant de boucle :
\begin{center}
P : \og si \pythoninline{val} est dans \pythoninline{liste_triee} son indice ne peut-être qu'entre \pythoninline{gauche} et \pythoninline{droite}\fg{}
\end{center}

\begin{itemize}
    \item   avant l'entrée dans la boucle \pythoninline{while}, on a\\
     \pythoninline{gauche = 0} et \pythoninline{droite = len(liste_triee)-1} donc P est trivialement vérifiée;
    \item   dans la boucle, si  \pythoninline{liste_triee[milieu] == val} alors on renvoie \pythoninline{val} et la fonction s'arrête et donne bien le résultat attendu;
    \item   sinon si \pythoninline{liste_triee[milieu] > val} alors puisque la liste est triée, la position de \pythoninline{val} ne peut être qu'entre \pythoninline{gauche} et \pythoninline{milieu-1}, or \pythoninline{droite} est actualisée avec cette valeur, et P reste vraie.
    \item   de même si \pythoninline{liste_triee[milieu] < val}
\end{itemize}
En sortie de boucle on n'a plus la condition \\
\pythoninline{gauche <= droite} donc on a \pythoninline{gauche > droite}. Supposons que \pythoninline{val} appartienne à la liste, alors puisque l'invariant de boucle P est vrai y compris à la fin de la boucle, cela veut dire que \pythoninline{val} doit être entre \pythoninline{gauche} et \pythoninline{droite} mais c'est absurde puisque\\  \pythoninline{gauche > droite}. Donc notre supposition est fausse : \pythoninline{val} n'appartient pas à la liste\footnote{Ceci s'appelle un \textit{raisonnement par l'absurde.}}.\\

On a donc prouvé la correction de notre fonction.

\section{Complexité}

On va ici évaluer le nombre d'étapes nécessaires au déroulement de la fonction. On va raisonner dans le pire des cas : \pythoninline{val} n'appartient pas à la liste.

Sans détailler les calculs\footnote{Peux-tu le faire ?}, à chaque itération de boucle, \pythoninline{droite-gauche+1} (nombre de valeurs qui restent à examiner) est au moins divisé par 2 et lorsque  cette valeur vaut 1, c'est qu'on est à la dernière itération de boucle : \pythoninline{droite == gauche} et on est sûr ou bien de trouver \pythoninline{val} à cet endroit, ou bien on sort de la boucle.\\

Ainsi, pour une liste de longueur 2 on est sûrs d'arriver au résultat en 2 itérations, pour une liste de longueur 4, en 3 itérations et en généralisant, si la liste est de longueur $2^n$, en $n+1$ itérations.\\

Par exemple pour un tableau de longueur 1000, puisque $2^9<1000<2^{10}$, on est sûr d'arriver au résultat au plus en 10 itérations.\\

\begin{definition}[]
Soit $n$ un entier naturel non nul, on appelle \textit{logarithme en base 2} de $n$ l'unique réel $x$ solution de $$2^x=n$$
Ce nombre $x$ est noté $\log_2(n)$.

% Le nombre de bits nécessaires pour écrire $n$ en binaire est $$\lfloor\log_2 n\rfloor +1$$
\end{definition}

Ce que l'on vient de prouver, c'est que pour une liste de taille $n$, la fonction\\
 \pythoninline{recherche_dichotomique} nécessitera au plus $E(\log_2(n))+1$ itérations pour déterminer si oui ou non une valeur appartient à cette liste ($E$ représente la fonction \textit{partie entière}).\\
 
\begin{propriete}
Soit une liste triée de longueur $n\in\N^*$.\\
Soit $p$ le nombre de bits nécessaires pour écrire $n$ en base 2.\\

La recherche dichotomique d'une valeur dans la liste nécessite \textbf{au plus} $p$ accès à cette liste.
\end{propriete}

Pour cette raison la complexité de l'algorithme de recherche dichotomique est dite \textit{logarithmique}. C'est bien mieux que celle de la recherche simple.
\chapter{Algorithmes gloutons}
\section{Une manière de procéder...}

\begin{definition}[ : algorithme glouton]
Un algorithme est dit \textit{glouton} lorsque
\begin{itemize}
	\item 	il procède étape par étape, avec une boucle ;
	\item 	à chaque itération il essaye d'\textit{optimiser} une grandeur (maximiser ou minimiser) en faisant un \textit{choix} ;
	\item 	les choix faits sont \textit{définitifs} : ils ne sont jamais remis en questions lors des itérations suivantes.	
\end{itemize}
\end{definition}

\begin{exemple}[ : rendu de monnaie]
Lorsqu'on rend la monnaie en euros et qu'on veut rendre le moins de pièces (ou billets) possibles, on
\begin{itemize}
\item 	procède pièce par pièce ;
\item  	choisit la pièce dont la valeur est la plus grande possible tout en restant inférieure ou égale au montant qu'il reste à rendre ;
\item 	on continue ainsi jusqu'à ce qu'il ne reste plus rien à rendre, sans jamais reprendre une pièce rendue auparavant.
\end{itemize}
Cette méthode est gloutonne et elle permet toujours de rendre la monnaie avec le moins de pièces possible (en tout cas lorsque le système monétaire est l'euro).
\end{exemple}
\section{Qui n'est pas toujours optimale}


Considérons un robot placé en A, qui veut monter le plus haut possible.
S'il applique la méthode gloutonne suivante :\medskip\par
\floatpictureright{0.33}{ch-gloutons/img/glouton}{
\begin{itemize}
\item à chaque seconde, tant que possible ;
\item regarder à droite ou à gauche sur une petite distance ;
\item aller dans la direction ou la pente est la plus forte.
\end{itemize}
}\medskip\par
Alors il se retrouvera en m, et pas en M.

\begin{aretenir}
\begin{itemize}
\item  un algorithme glouton ne fournit pas toujours une solution optimale ;
\item pour s'assurer qu'il fournit une démonstration optimale, il faut le \textit{démontrer}.
\end{itemize}
\end{aretenir}

\section{Des exemples optimaux}

\begin{itemize}
	\item le rendu de pièces en euros ;
	\item l'écriture d'un entier naturel en binaire par la méthode des soustractions (qui correspond à un rendu de pièces qui ont des valeurs de $2^n$) ;
	\item l'algorithme dit « des conférenciers » ;	
\end{itemize}
Nous en verrons d'autres en Terminale.

\section{Des exemples non optimaux}

\begin{itemize}
	\item le problème du robot exposé précédemment ;
	\item les méthodes gloutonnes pour résoudre le problème du « sac à dos ».
\end{itemize}


\chapter{L'algorithme des k plus proches voisins}
Cet algorithme s'appelle $k$ \textit{Nearest Neighbors} en anglais, nous l'appellerons donc $k$NN.

\section{Des questions concrètes}

\begin{enumerate}
	\item Une entreprise de vente d'articles en lignes collecte les informations de ses clients. Elle en a accumulé un grand nombre, tels que 
	\begin{itemize}
		\item l'âge ;
		\item le sexe ;
    	\item l'adresse ;
    	\item les revenus moyens mensuels ;
    	\item \textit{et cætera}.
	\end{itemize}
    En fonction des achats, réguliers ou non, elle a placé ses client.e.s dans des catégories telles que
    \begin{itemize}
    	\item client\cdot e fidèle ;
    	\item client\cdot e à fidéliser ;
    	\item \textit{et cæetera}.
    \end{itemize}
    Un nouveau client se présente.\\
    Connaissant son âge, son sexe, son adresse et ses revenus mensuels, dans quelle catégorie l'entreprise va-t-elle le placer ?

	\item On a mesuré la largeur et la longueur des pétales et des  sépales d'iris de 3 catégories (\textit{iris setosa}, \textit{iris versicolor} et \textit{iris virginica}) 
    On effectue des mesures sur une nouvelle fleur. Dans quelle catégorie va-t-on la placer ? 

\end{enumerate}

Pour répondre aux deux questions précédentes on utilise le même algorithme : $k$NN.


\section{Un algorithme pour y répondre}

\floatpictureleft{0.33}{ch-knn/img/plot}{On considère deux nuages de points, l'un vert et l'autre bleu.
Le point rouge (appelons-le P) doit-il être considéré comme appartenant au nuage vert ou au nuage bleu ?\\

Pour répondre à cette question, on va }\medskip\par



\begin{methode}[ : kNN]
\begin{itemize}
	\item choisir un entier $k$ impair (3, 5 ou 7 typiquement) ;
	\item calculer les distances entre P et tous les autres points ;
	\item sélectionner les $k$ points les plus proches de P ;	
	\item regarder leurs couleurs.
\end{itemize}
Puisque $k$ est impair il n'y aura pas de situation d'\textit{ex æquo} et, suivant la couleur majoritaire des « $k$ plus proches voisins » de P, on pourra choisir celle de P.\\

C'est cela, l'algorithme $k$NN.
\end{methode}

La seule chose qui change selon la situation, c'est la \textit{distance} que l'on utilise. Avec les nuages de points, on utilise la distance euclidienne :
$$d(A,B)=\sqrt{\left(x_B-x_A\right)^2+\left(y_B-y_A\right)^2}$$
Mais on peut utiliser une distance différente suivant la situation.\\
Par exemple, si on ne manipule non plus des points mais des quadruplets $A=(a_0,a_1,a_2,a_3)$, alors on peut poser

$$d(A,B)=\sqrt{(b_0-a_0)^2+(b_1-a_1)^2+(b_2-a_2)^2+(b_3-a_3)^2}$$

ou encore

$$d(A,B)=|b_0-a_0|+|b_1-a_1|+|b_2-a_2|+|b_3-a_3| $$

Et d'ailleurs, si c'est la proximité de la première composante qui importe le plus, on peut définir
$$d(A,B) =50 \times |b_0-a_0|+|b_1-a_1|+|b_2-a_2|+|b_3-a_3|$$
Pour plus de renseignements sur ce qu'est une distance, tu peux consulter Wikipédia.


\tableofcontents
\end{document}