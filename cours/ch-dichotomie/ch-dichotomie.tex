
\chapter{Recherche dichotomique}
\introduction{Plus petit ou plus grand ?}

\section{Présentation de l'algorithme}

Lorsqu'on cherche si un élément appartient ou non à une liste, il suffit de la parcourir en comparant chacun de ses éléments à celui que l'on cherche.

Cette démarche peut être améliorée si la liste possède des propriétés particulières, notamment si c'est une liste d'entiers triée.


On veut écrire une fonction \texttt{recherche\_dichotomique} qui :

En entrée prend \begin{itemize}
	\item 	une liste \mintinline{python}{liste_triee} de $n$ entiers \textit{triée dans l'ordre croissant};
	\item 	un entier \mintinline{python}{val}.
\end{itemize}
Renvoie \begin{itemize}
	\item 	l'indice de \mintinline{python}{val} dans \mintinline{python}{liste_triee} si \mintinline{python}{val} appartient à \mintinline{python}{liste_triee};
	\item 	-1 si a n'appartient pas à \mintinline{python}{liste_triee}.
\end{itemize}



\begin{exemple}[]
	\begin{itemize}
		\item 	\mintinline{python}{recherche_dichotomique([11, 20, 32, 33, 54], 32)} \\renvoie 2 car 32 est l'élément d'indice 2 de la liste.
		\item 	\mintinline{python}{recherche_dichotomique([20, 32, 33, 54], 40)} \\renvoie -1 car 40 ne figure pas dans la liste.
	\end{itemize}
\end{exemple}

\begin{methode}[]
	On compare \mintinline{python}{val} avec l'élément \mintinline{python}{m} qui se situe « à peu près au milieu de \mintinline{python}{liste_triee} ».
	\begin{itemize}
		\item 	si \mintinline{python}{val} est égal à \mintinline{python}{m} c'est gagné, on renvoie l'indice de \mintinline{python}{m} dans  \mintinline{python}{liste_triee};
		\item 	sinon si \mintinline{python}{val > m} on recommence avec la liste des éléments situés après \mintinline{python}{m}.
		\item 	sinon c'est que \mintinline{python}{val < m} et on recommence avec la liste des éléments situés avant \mintinline{python}{m}.
	\end{itemize}
	On itère ce procédé tant que la liste des valeurs à examiner n'est pas vide. Si on arrive à une liste vide c'est que \mintinline{python}{val} , n'est pas dans \mintinline{python}{liste_triee}.
\end{methode}
Voici l'algorithme traduit en \textsc{Python} :
\begin{pyc}
			\begin{minted}{python}
01	def recherche_dichotomique(liste_triee, val):
02		# début de la plage de valeurs à regarder
03		gauche = 0  
04		# fin de la plage
05		droite = len(liste_triee) - 1
06		# tant que la plage est non vide  
07		while gauche <= droite:  
08			# on prend grosso modo le milieu
09			milieu = (gauche + droite) // 2  
10			# si on trouve val au milieu c'est gagné
11			if liste_triee[milieu] == val:
12				return milieu  
13			# si on a dépassé val
14			elif liste_triee[milieu] > val:  
15				# alors on regarde avant
16				droite = milieu - 1  
17			# sinon on regarde après
18			else:
19				gauche = milieu + 1  
20		# si on est sorti de la boucle
21		# c'est qu'on n'a pas trouvé val
22		return -1  
	\end{minted}
\end{pyc}

\section{Comprendre l'algorithme}

Commençons par le faire « tourner à la main » sur un exemple, avec\\ \mintinline{python}{liste_triee} valant \mintinline{python}{[1, 3, 4, 8, 9, 13, 20, 21]} .\\

On cherche la valeur 4.\\

Pour chaque itération on a noté dans un tableau les valeurs des variables (celles de \mintinline{python}{gauche} et \mintinline{python}{droite} avant qu'elles ne soient modifiées) et si la fonction renvoie quelque chose ou non.
\tabstyle[UGLiOrange]
\begin{center}
	\small
	
	\begin{tabular}{|c|c|c|c|c|c|}
		\hline
		\ccell   n° d'itér & {\ccell   gauche} & {\ccell   droite} & {\ccell   milieu} & {\ccell   liste\_triee[milieu]} & {\ccell   valeur renvoyée} \\
		\hline
		1                                                     & 0                             & 7                             & 3                             & 8                                           & NON                                    \\
		\hline
		2                                                     & 0                             & 2                             & 1                             & 3                                           & NON                                    \\
		\hline
		3                                                     & 2                             & 2                             & 2                             & 4                                           & OUI : 2                                \\
		\hline
	\end{tabular}
\end{center}
Ainsi la fonction a renvoyé 2, indice de la valeur 4 dans la liste, au bout de 3 itérations.\\

On cherche la valeur 15:
\begin{center}

	\begin{tabular}{|c|c|c|c|c|c|}
		\hline
		{\ccell  n°itér} & {\ccell  gauche} & {\ccell  droite} & {\ccell  milieu} & {\ccell  liste\_triee[milieu]} & {\ccell  return ?} \\
		\hline
		1                                                  & 0                             & 7                             & 3                             & 8                                           & NON                             \\
		\hline
		2                                                  & 4                             & 7                             & 5                             & 13                                          & NON                             \\
		\hline
		3                                                  & 6                             & 7                             & 2                             & 20                                          & NON                             \\
		\hline
		-                                                  & 7                             & 6                             & -                             & -                                           & OUI : -1                        \\
		\hline
	\end{tabular}
\end{center}
La dernière ligne du tableau signifie qu'au bout de la 3\eme itération, les conditions de boucles ne sont plus vérifiées (car \mintinline{python}{gauche > droite}) et que -1 est renvoyé.
\begin{exercice}

	On cherche la valeur 6 dans la liste précédente.

	Complète le tableau (des lignes resteront peut-être vides).
	\begin{center}
		\begin{tabular}{|c|c|c|c|c|c|}
			\hline
			{\ccell  n°itér} & {\ccell  gauche} & {\ccell  droite} & {\ccell  milieu} & {\ccell  liste\_triee[milieu]} & {\ccell  return ?} \\
			\hline
			                                                   &                               &                               &                               &                                             &                                 \\
			\hline
			                                                   &                               &                               &                               &                                             &                                 \\
			\hline
			                                                   &                               &                               &                               &                                             &                                 \\
			\hline
			                                                   &                               &                               &                               &                                             &                                 \\
			\hline
		\end{tabular}
	\end{center}

	On cherche la valeur 21, complète le tableau (des lignes resteront peut-être vides).
	\begin{center}
		\begin{tabular}{|c|c|c|c|c|c|}
			\hline
			{\ccell  n°itér} & {\ccell  gauche} & {\ccell  droite} & {\ccell  milieu} & {\ccell  liste\_triee[milieu]} & {\ccell  return ?} \\
			\hline
			                                                   &                               &                               &                               &                                             &                                 \\
			\hline
			                                                   &                               &                               &                               &                                             &                                 \\
			\hline
			                                                   &                               &                               &                               &                                             &                                 \\
			\hline
			                                                   &                               &                               &                               &                                             &                                 \\
			\hline
		\end{tabular}
	\end{center}
\end{exercice}
\section{Analyse de l'algorithme}

Quatre questions se posent :

\begin{enumerate}
	\item 	Pourquoi, lorsque la fonction renvoie un entier positif, est-ce bien la position de \mintinline{python}{val} dans \mintinline{python}{liste_triee} ?
	      C'est un problème de \textit{correction}.
	\item 	Quand la fonction renvoie -1, est-ce que cela veut bien dire que \mintinline{python}{val} n'est pas dans \mintinline{python}{liste_triee} ?
	      C'est un problème de \textit{complétude}.
	\item 	Pourquoi la boucle \textit{tant que} s'arrête-t-elle toujours ? On dit que c'est un problème de \textit{terminaison}.
	\item	Enfin, pourquoi cette fonction est-elle plus rapide qu'un parcours des éléments un par un ?
	      C'est un problème de \textit{complexité}.
\end{enumerate}

\section{Correction de l'algorithme}

Quand la fonction renvoie un entier positif, c'est à la ligne 12  , ce qui signifie qu'on a effectivement trouvé \mintinline{python}{val} dans \mintinline{python}{liste_triee}, à la position renvoyée.  


\section{Complétude de l'algorithme}
Pour prouver que cette fonction est complète, on doit utiliser un \textit{invariant de boucle}.

\begin{definition}
	Un \textit{invariant de boucle} est une propriété $\mathcal{P}$ dépendant éventuellement des variables du programme. 
	\begin{itemize}
		\item 	$\mathcal{P}$ doit être vraie avant l'entrée dans la boucle ;
		\item 	$\mathcal{P}$ doit rester vraie à chaque itération de boucle ;
		\item 	à la fin de la boucle, $\mathcal{P}$ doit nous permettre de conclure que la fonction « fait bien ce qu'elle doit faire ».
	\end{itemize}
\end{definition}

Dans notre cas voici l'invariant de boucle :
\begin{center}
	$\mathcal{P}$ : « si \mintinline{python}{val} est dans \mintinline{python}{liste_triee} son indice est entre \mintinline{python}{gauche} et \mintinline{python}{droite}»
\end{center}

\begin{itemize}
	\item 	avant l'entrée dans la boucle \mintinline{python}{while}, on a\\
	      \mintinline{python}{gauche == 0} et \mintinline{python}{droite == len(liste_triee) - 1} donc P est trivialement vérifiée ;
	\item 	dans la boucle, si  \mintinline{python}{liste_triee[milieu] == val} alors on renvoie \mintinline{python}{val} et la fonction s'arrête et donne bien le résultat attendu ;
	\item   sinon si \mintinline{python}{liste_triee[milieu] > val} alors puisque la liste est triée, la position de \mintinline{python}{val} ne peut être qu'entre \mintinline{python}{gauche} et \mintinline{python}{milieu-1}, or \mintinline{python}{droite} est actualisée avec cette valeur, et P reste vraie ;
	\item 	de même si \mintinline{python}{liste_triee[milieu] < val} ;
	\item En sortie de boucle $\mathcal{P}$ est toujours vérifiée et puisque \mintinline{python}{gauche > droite} cela signifie que \mintinline{python}{val} n'est pas dans \mintinline{python}{liste_triee}.  
\end{itemize}

On a donc prouvé la complétude de notre fonction.

\section{Terminaison}

Pour prouver qu'une boucle \textit{tant que} se termine, \textit{en théorie} on détermine un \textit{variant} de boucle.

\begin{definition}[]
	Un variant de boucle est un \textit{entier positif qui décroît strictement à chaque itération de boucle}. On le choisit de sorte à ce que lorsqu'il atteint zéro (ou un, en tout cas une petite valeur) la boucle se termine.\\
\end{definition}

Dans notre cas, le variant de boucle est l'entier \mintinline{python}{v} défini par \mintinline{python}{v = droite - gauche} : la condition du \mintinline{python}{while} est liée à \mintinline{python}{v} puisque \mintinline{python}{gauche <= droite} équivaut à \mintinline{python}{v >= 0}.\\

Pour montrer que \mintinline{python}{v} décroît strictement il suffit de montrer que ou bien gauche augmente strictement ou bien droite décroît	strictement.

Or lors d'une itération, \mintinline{python}{m} est toujours entre \mintinline{python}{gauche} et \mintinline{python}{droite} (au sens large) et
\begin{itemize}
	\item soit on trouve que \mintinline{python}{liste_triee[m]} vaut \mintinline{python}{val} et la boucle s'arrête ;
	\item sinon ou bien \mintinline{python}{gauche} devient \mintinline{python}{m + 1} donc augmente strictement, ou bien \mintinline{python}{droite} devient \mintinline{python}{m - 1} donc décroît strictement.
\end{itemize}

Ainsi les valeurs de v décroissent strictement, donc finissent (si on ne trouve pas \texttt{val}) par atteindre zéro et la boucle se termine.\\
On dit qu'on a prouvé la \textit{terminaison} de la fonction.



\section{Complexité}

On va ici évaluer le nombre d'étapes nécessaires au déroulement de la fonction. On va raisonner dans le pire des cas : \mintinline{python}{val} n'appartient pas à la liste.

À chaque itération de boucle, le nombre de valeurs qui restent à examiner est au moins divisé par 2 et lorsque  cette valeur vaut 1, c'est qu'on est à la dernière itération de boucle et on est sûr ou bien de trouver \mintinline{python}{val} à cet endroit, ou bien on sort de la boucle et on renvoie \mintinline{python}{-1} .\\

Ainsi, pour une liste triée de taille $n$, le nombre d'itérations de la boucle dans le pire des cas, c'est le plus petit entier $k$ tel que $2^k$ dépasse $n$.\\

Pour une liste de longueur 2 on est sûrs d'arriver au résultat en 2 itérations, pour une liste de longueur 4, en 3 itérations et en généralisant, si la liste est de longueur $2^n$, en $n+1$ itérations.\\

Pour un tableau de longueur 1000, puisque $2^9<1000<2^{10}$, on est sûr d'arriver au résultat au plus en 10 itérations.\\

\begin{definition}[]
	Soit $n$ un entier naturel non nul, on appelle \textit{logarithme en base 2} de $n$ l'unique réel $x$ solution de $$2^x=n$$
	Ce nombre $x$ est noté $\log_2(n)$.

	% Le nombre de bits nécessaires pour écrire $n$ en binaire est $$\lfloor\log_2 n\rfloor +1$$
\end{definition}

Ce que l'on vient de prouver, c'est que pour une liste de taille $n$, la fonction\\
\mintinline{python}{recherche_dichotomique} nécessitera au plus $E (\log_2(n))+1$ itérations pour déterminer si oui ou non une valeur appartient à cette liste ($E$ représente la fonction \textit{partie entière}).\\

\begin{propriete}
	Soit une liste triée de longueur $n\in\N^*$.\\
	Soit $p$ le nombre de bits nécessaires pour écrire $n$ en base 2.\\

	La recherche dichotomique d'une valeur dans la liste nécessite \textbf{au plus} $p$ accès à cette liste.
\end{propriete}

Pour cette raison la complexité de l'algorithme de recherche dichotomique est dite \textit{logarithmique}. C'est bien mieux que celle de la recherche simple.

\begin{exercice}[ : efficacité de l'algorithme]
	\begin{enumerate}
		\item Dans une liste triée de taille \np{10000}, en combien d'étapes l'algorithme de recherche dichotomique s'arrête-t-il \textit{dans le pire des cas} ? 
		\item Même question pour une liste de taille \np{100000} et pour une liste de taille \np{1000000}.
	\end{enumerate}
	
\end{exercice}
